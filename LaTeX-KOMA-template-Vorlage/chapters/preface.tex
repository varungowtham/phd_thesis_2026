
\addchap*{Preface/Geleitwort/Vorwort}
\label{cha:preface}

Das Geleitwort \enquote{preface} kann der Autor der Arbeit nutzen, um etwas über die Entstehungsgeschichte der Arbeit zu schreiben.
Im Gegensatz dazu muss das \enquote{foreword} nicht unbedingt von Autor selbst stammen -- man kennt das von Büchern, die gerne mit \enquote{mit einem Vorwort von \dots} beworben werden. Dort, also im \enquote{foreword}, wird dem Leser eine Motivation gegeben, warum er die Arbeit lesen sollte.
Da in wissenschaftlichen Arbeiten die Kurzfassung bereits die Rolle der Motivation übernimmt, bietet es sich an, das Vorwort im Sinne eines Editorials/Geleitworts zu interpretieren und über die Begleitumstände der vorliegenden Arbeit zu schreiben. Wenn bereits eine Danksagung Teil der Arbeit ist, sollte man besser darauf verzichten, diese im Geleitwort zu wiederholen. Andernfalls wäre hier durchaus der richtige Ort dafür.

Diese Arbeit entstand im Sommer 2017, als Prof. Repke -- genervt von überlangen, schlecht lesbaren und mit mangelhaftem Literaturverzeichnis und fragwürdigem Zitierstil ausgestatteten Abschlussarbeiten -- mich beauftragt hat, eine Richtlinie für Abschlussarbeiten am Fachgebiet \enquote{Dynamik und Betrieb technischer Anlagen} (DBTA) zu formulieren.
Basierend auf Vorgaben des Prüfungsamts der TU-Berlin, einem Buch zum Thema Abschlussarbeiten, einer Vorlage für Dissertationen, und Richtlinien, mit denen verschiedene aktuelle Mitarbeiter schon in Kontakt kamen, habe ich eine Vorlage erstellt, mit der unter anderem genau dieses Dokument erstellt wurde.

Möge diese Vorlage noch lange von Nutzen sein, Korrektoren die Arbeit erleichtern, Studierenden und Promovierenden Struktur- und Formatierungsentscheidungen abnehmen und Lesern Vorfreude auf weitere Abschlussarbeiten aus unserem Fachgebiet bereiten.
Auch wenn das Vorwort weniger formell ist als der Haupttext, sollte man auch hier auf unangemessene Elemente wie Emoticons verzichten. So bleibt eigentlich nur noch zu hoffen, dass der Inhalt mit der Form mithalten kann ;-)\\


Berlin, Juli 2017 \hspace{6cm} \textit{Gregor Tolksdorf}



