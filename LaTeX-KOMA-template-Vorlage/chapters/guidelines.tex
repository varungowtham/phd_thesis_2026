
\chapter{Richtlinien/Guidelines für Abschlussarbeiten}
\label{cha:guidelines}

\section[Allgemein]{Allgemein}
\begin{enumerate}
	\item Die aktuell geltende Studien- und Prüfungsordnung ist maßgeblich. Folgende Regelungen greifen nur, wenn sie den geltenden Ordnungen nicht widersprechen.
	\item 	Masterarbeiten sind wissenschaftlich-technische Dokumentationen, die allgemeinen Anforderungen bezüglich Struktur und Form entsprechen müssen. Sie sollen sich durch Klarheit im Ausdruck, guten Stil und einwandfreie Orthografie auszeichnen. Formulierungen sind sorgfältig zu wählen.
\item Die Arbeit ist logisch zu gliedern.
\item Die Abschlussarbeit soll einen anspruchsvollen wissenschaftlich-technischen Inhalt verständlich vermitteln. Dazu ist es erforderlich, sich immer wieder in die Position des Lesers zu versetzen und die Darstellung in dieser Hinsicht zu hinterfragen.
\item Das Bild, d.h. Prinzipskizze, Diagramm, Foto, Flussdiagramm, Tabelle etc., ist die „Sprache das Ingenieurs“ und sollte langen Erklärungen vorgezogen werden.
\item Ergebnisse müssen rückverfolgbar sein. Dazu sind die angewendeten Methoden, Annahmen, Randbedingungen, experimentellen Einrichtungen und Programme sowie relevante Zwischenergebnisse zu nennen und in einem angemessenen Umfang zu erläutern.
\item Berechnungen sind so zu dokumentieren, dass der Gutachter ihre Richtigkeit überprüfen kann.
\item Die Darstellung sollte sich auf das Wesentliche konzentrieren und frei von allgemein bekannten Abhandlungen sein, die nur vom Thema ablenken und den „roten Faden“ verlieren lassen.
\end{enumerate}


\section[Form]{Form}
\begin{enumerate}
	\item Die Abschlussarbeit ist in der Regel in deutscher Sprache und dabei nach den Regeln der neuen deutschen Rechtschreibung anzufertigen. 
\item Sätze müssen klar verständlich sein. Schachtelsätze sollten bewusst und nicht zu oft formuliert werden.
\item Der Unterschied zwischen Stichpunkten und Sätzen ist zu beachten.
\item Physikalische Größen sind gemäß DIN 1304-1 und ENISO 80000 in Maßeinheiten des internationalen Einheitensystems (SI) anzugeben. Insbesondere stehen Einheiten niemals in eckigen Klammern.
\item Formelzeichen und Abkürzungen sind in einem gesonderten Verzeichnis in alphabetischer Reihenfolge oder, bei nur einmaliger Benutzung, im Text zu erläutern. Die physikalischen Einheiten sind neben Symbol und Bedeutung in einer separaten Spalte des Symbolverzeichnisses aufzuführen.
\item Bilder, Tabellen und Formeln sind zu nummerieren und müssen einen Textbezug haben. Bilder werden mit einer Bildunterschrift versehen.
\item Bilder, die in den Text eingefügt werden, sollten sorgfältig ausgewählt werden, um Textfluss und Lesbarkeit zu unterstützen und nicht zu beeinträchtigen. Die wesentlichen Details des Bildes und relevante Beschriftungen müssen in allen abgegebenen Exemplaren gut erkennbar sein. 
\item Gleichungen sind fortlaufend je Kapitel zu nummerieren, die Nummer des Kapitels wird, durch einen Punkt abgetrennt, vorangestellt. Die Gleichungsnummer ist in runden Klammern anzugeben und rechtsbündig zu platzieren. Beispielsweise ist (3.1) die erste Gleichung in Kapitel 3, wohingegen (B.13) die dreizehnte Gleichung in Anhang B ist.
\item Erläuternde Beispiele, große Bilder und ergänzende Daten sowie Beilagen, auf die im Text immer wieder hingewiesen wird, gehören in die Anlage. Längere Dokumente, die der Verständlichkeit dienen, Programmquelltexte, längere Beispiele o.ä. sind als Anhang in gedruckter oder elektronischer Form beizufügen.
\item Enthält die Arbeit Zitate, so müssen diese entsprechend ausgewiesen und die Quellen nachgewiesen werden. Es muss klar ersichtlich sein, welche Erkenntnisse vom Autor stammen und welche der Literatur entnommen wurden. Alle benutzten Quellen, auch für sinngemäß wiedergegebene Gedanken, Erfahrungswerte usw. sind im Text zu referenzieren und im Literaturverzeichnis aufzuführen. Wörtliche Zitate sind durch Anführungszeichen zu kennzeichnen.
\end{enumerate}


\section[Formatierung]{Formatierung}
\begin{enumerate}
	\item Die Arbeit ist in einer proportionalen Serifenschrift zu verfassen, \zB Palatino in LaTeX (mathpazo). Überschriften dürfen, wenn die Arbeit mit LaTeX erstellt wird, serifenlos sein.
\item Der Text ist einseitig im Format A4 in Blocksatz und mit einer Schriftgröße von mindestens 11 pt zu setzen. 
\item Das Textfeld (\enquote{Satzspiegel}) muss eine Breite zwischen 130 mm und 155 mm haben, wodurch sich Ränder von 30-40 mm je Seite ergeben. In der ausgedruckten Version ist eine angemessene Bindekorrektur einzuplanen, die den Satzspiegel um rund 5-10 mm nach rechts verschiebt (und somit den linken Rand gegenüber dem rechten Rand vergrößert).
\item Der Satzspiegel nimmt in der Höhe zwischen 6/9 und 8/11 der Seite ein, wobei der Abstand zum unteren Rand der Seite doppelt so groß ist wie der Abstand zum oberen Rand ist. Das entspricht Rändern von 27 - 33 mm (oben) und 54 - 66 mm (unten).
\item Die Seitenzahl befindet sich zentriert unter dem Satzspiegel mit mindestens zwei Zeilen Abstand von diesem.
\item Über dem Satzspiegel werden, mit mindestens einer Zeile Abstand, in serifenloser Schrift die Kapitelnummer und die Kapitelüberschrift in gleicher Schriftgröße wie der Haupttext platziert, z.B. \enquote{4. Ergebnisse}. Diese \enquote{Erinnerungsüberschriften} dürfen durch eine durchgezogene Linie vom Satzspiegel optisch abgehoben werden.
\item Überschriften sind genauso wie Legenden zu Abbildungen sowie Tabellen linksbündig zu setzen. Der Abstand vor/über einer Überschrift ist 1,5- bis 2-mal so groß wie danach/darunter (zu Text oder Unterüberschrift).
\item In Bildunterschriften etc. ist eine kleinere Schriftgröße zu wählen (z.B. 9 pt statt 11 pt).
\item Jede Seite im Hauptteil der Arbeit hat als Daumenregel maximal 74 Zeichen pro Zeile und 34 Zeilen pro Seite. Das entspricht maximal rund 2500 Zeichen pro Seite.
[In LaTeX (KOMA-Script) ist für die Schrift Palatino ein linespread von 1.2 vorzusehen (Spielraum: 1.0 bis 1.5). Das kann in anderen Sprachen einem Zeilenabstand von 1,2 bis 1,5 entsprechen. Die maximale Anzahl an Zeilen gibt hier eine Orientierung.]
\item Der gesamte Hauptteil der Arbeit soll nicht mehr als 220.000 Zeichen enthalten (das entspricht knapp 90 Seiten). Wird diese Grenze überschritten, ist gemeinsam mit dem Betreuer zu prüfen, ob eine Kürzung angebracht/erforderlich ist. 
\end{enumerate}


\section[Erscheinungsbild]{Äußeres Erscheinungsbild}
\begin{enumerate}
	\item Im Hauptteil (\enquote{mainmatter}) beginnt die Zählung der Seiten mit arabischen Ziffern bei 1, vorher (\enquote{frontmatter}) sind römische Ziffern zu verwenden.
\item Es sind maximal vier Gliederungsebenen zu verwenden und maximal drei Ebenen im Inhaltsverzeichnis zu notieren.
\item 	Die einzelnen Abschnitte des Anhangs sind nicht numerisch, sondern durch Großbuchstaben zu kennzeichnen.
\item Zur Hervorhebung im Text sind bevorzugt Fett- oder Kursivschreibung zu verwenden, Unterstreichungen sind nicht erwünscht.
\item Absätze, die auf zwei Seiten verteilt sind, dürfen nicht in der letzten oder vorletzten Zeile einer Seite beginnen (\enquote{Schusterjunge}) oder in der ersten oder zweiten Zeile einer Seite enden (\enquote{Hurenkinder}).
\end{enumerate}


\section[Quellenangaben und Literaturverzeichnis]{Quellenangaben und Literaturverzeichnis}
Zitate/Quellenangaben dienen dazu,
\begin{itemize}
	\item eigene Aussagen zu belegen und zu begründen,
	\item eigene Aussagen von fremden zu unterscheiden,
	\item den Lesern die Quellen der Aussagen rasch nachvollziehbar zu machen.
\end{itemize}
Alle fremden Inhalte werden im Text mit einem kurzen Quellenhinweis gekennzeichnet, der durch eine ausführlichere Quellenangabe im Literaturverzeichnis ergänzt wird. Dabei spielt es keine Rolle, ob die fremden Inhalte wörtlich/direkt oder sinngemäß/indirekt übernommen werden.\\
Jede Quellenangabe im Text weist auf eine Angabe im Literaturverzeichnis hin und jede Quellenangabe im Literaturverzeichnis ist die Erläuterung (mindestens) eines Kurzhinweises im Text. Wenn die Bibliographie mit LaTeX erstellt wird, ergibt sich das quasi von selbst. Es ist erlaubt, im Quellenverzeichnis die Seiten der Vorkommen im Text zu nennen und Referenzen (Links) zu setzen.\\
Für Abschlussarbeiten verwenden wir bevorzugt das \enquote{Namen-Datum-/Autor-Jahr-System}, auf das sich die folgenden Erläuterungen und Beispiele beziehen; das \enquote{Nummernsystem} ist aber auch möglich. Bedeutendste Grundregel von allen: Vollständige und einheitliche Angaben sind wichtiger als das Einhalten eines speziellen Standards. \\
Die Mindestanforderungen sind: Autor, Titel, Datum/Jahr.

\subsection{Zitieren im Literaturverzeichnis}

Wenn LaTeX zur Erstellung der Abschlussarbeit verwendet wird, ist bevorzugt der Biblatex-Stil \enquote{authoryear-comp} (oder \enquote{authoryear}) zu wählen. \textit{Alle folgenden Angaben sind also nur dann relevant, wenn das Literaturverzeichnis von Hand erstellt wird.}\\

\subsubsection{Literaturverzeichnis von Hand anlegen}
Die Quellen werden alphabetisch nach dem Nachnamen ihrer Verfasser sortiert. Sollten mehrere Werke von denselben Autoren stammen, so werden diese Quellenangaben chronologisch gereiht (von alten nach jungen Veröffentlichungen). Wenn dies noch nicht eindeutig ist, wird die Jahreszahl zusätzlich mit einem Kleinbuchstaben versehen.
Innerhalb der einzelnen Literaturangaben wird die Reihenfolge der Autoren unverändert aus der Originalquelle entnommen. Die ersten sechs Autoren werden immer angegeben; alle weiteren Autorennamen werden durch \enquote{et al.} ersetzt. Bei zwei bis sechs Autoren darf vor dem letzten Autor ein \enquote{und}, \enquote{and} oder \enquote{\&} stehen.
Die Auflistung der Autoren wird durch einen Punkt abgeschlossen; danach folgt, durch ein Leerzeichen getrennt, das Jahr der Veröffentlichung (und ggflls. ein Kleinbuchstabe, mit oder ohne Leerzeichen von der Jahreszahl getrennt).
Nach dem Erscheinungsjahr folgen ein Doppelpunkt, ein Leerzeichen und der vollständige Haupttitel der Quelle (kursiv). Der Titel wird wiederum mit einem Punkt abgeschlossen.
Es folgen, je nach Art der Quelle, Angaben zu 
\begin{itemize}
	\item Zeitschrift/Sammelband (\enquote{In:}), 
	\item Verlag (\enquote{Burg-Verlag}), 
	\item Ort der Veröffentlichung (\enquote{Berlin}), 
	\item Jahrgang des Bandes, Heftnummer
	\item Seitenangaben (\enquote{182-188}),
	\item ISSN/ISBN/doi-Nummern (\enquote{ISSN 0815-0008})
\end{itemize}
Bei Artikeln in Zeitschriften oder in Sammelbänden müssen die Seitenzahlen angegeben werden.
Bei Büchern sollte neben dem Ort auch der Verlag angegeben werden. Bei Zeitschriften werden Verlag und Ort in der Regel nicht angegeben.
Wichtig ist die Einheitlichkeit der Form bei allen Angaben und Satzzeichen.
Die Nachnamen werden in Großbuchstaben geschrieben, von den Vornamen wird jeweils nur der erste Buchstabe genannt (ohne Abkürzungspunkt). 
Zwischen Nachname und Vorname steht ein einzelnes Leerzeichen, kein Komma.
Mehrere Verfasser werden mit Komma getrennt.
