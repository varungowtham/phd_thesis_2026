\chapter{Guidelines}
%------------------------------
% SEC: FIRST STEPS
%------------------------------
	\section{First Steps}
		\begin{itemize}
			\item \textbf{carefully read this whole chapter}
			\item check whether you can compile this document without errors (this should always be the case as long as all necessary packages are installed)
			\item change the necessary entries in meta.tex (i.e. your name, matriculation number, etc.)
			\item add your own .bib file for your references or use the present one
			\item check whether biblatex and makeindex were set up correctly and are operational (see Sections \ref{subsec:literature} and \ref{subsec:makeindex})
			\item set the language of the document (see Section \ref{sec:language})
			\item remove the guidelines chapter from this document by deleting it and removing it from the 0\_Text.tex file in the folder 03\_Content
			\item start writing your thesis - good luck!
		\end{itemize}
%------------------------------
% SEC: TEMPLATE STRUCTURE
%------------------------------		
	\section{Template Structure}
		This section describes the use of this template. Questions remaining unanswered can be forwarded to Christian Hoffmann (\href{mailto:c.hoffmann@tu-berlin.de}{c.hoffmann@tu-berlin.de}).
		\subsection{Main Features}
			\begin{itemize}
				\item can be used in German and English
				\item simple generation of list of symbols for latin, greek, etc.~with the nomentbl package
				\item simple generation of list of abbreviations with the acronym package
				\item includes DOI/ISBN/URL automatically in references
				\item includes bibliography via biblatex and biber
			\end{itemize}
		\subsection{Main Document}
			\begin{itemize}
				\item is called main.tex
				\item this document must be executed in LaTeX
			\end{itemize}
		\subsection{01\_Document\_administration}
			In this folder, the following files are located:
			\begin{enumerate}
				\item a\_Packages.tex
				\begin{itemize}
					\item contains all packages, which are loaded
					\item packags are sorted based on their application
				\end{itemize}
				\item b\_Commands.tex
				\begin{itemize}
					\item contains further commands regarding formating and look of the document 
					\item also structured
				\end{itemize}
				\item c\_Meta.tex
				\begin{itemize}
					\item contains meta information regarding author, etc.
					\item contains a boolean variable to select whether it is a dissertation or a bachelor/master thesis
				\end{itemize}
				\item d\_NomenclatureCommands.tex 
				\begin{itemize}
					\item defines structure of the list of symbols
					\item optional argument defines the class of a symbol (latin, greek, \dots)
					\item for the generation of the list of symbols, makeindex is used. A short instruction how to run makeindex correctly in texmaker is given below
				\end{itemize}
				\item e\_Header\_Footer.tex
				\begin{itemize}
					\item defines headers and footer
				\end{itemize}
			\end{enumerate} 
		\subsection{02\_Prematter}
			In this folder, the following files are located:
			\begin{enumerate}
				\item a\_Cover.tex
				\begin{itemize}
					\item creates the cover page
					\item loads either a\_Cover\_BA\_MA.tex (cover of a bachelor/master thesis) or a\_Cover\_Diss.tex (cover of a dissertation), depending on the value of type set in c\_Meta.tex
				\end{itemize}
				\item b\_Declaration.tex
				\begin{itemize}
					\item declaration that the thesis was written honestly
				\end{itemize}
				\item c\_Acknowledgements.tex
				\begin{itemize}
					\item thank important people
				\end{itemize}
				\item d\_Summary.tex
				\begin{itemize}
					\item summarize your thesis in German and English
				\end{itemize}
				\item e\_Nomenclature.tex
				\begin{itemize}
					\item enter all symbols and explain them
					\item examples are provided
				\end{itemize}
				\item f\_Abbreviations.tex
				\begin{itemize}
					\item enter all abbreviations and explain them
					\item examples are provided
				\end{itemize}
			\end{enumerate} 
		\subsection{02a\_Dissertation\_files}
			This folder contains files that are only needed for dissertations. Bachelor/Master theses do not have to consider them. However, if you want to dedicate your thesis to someone, you can of course include this file. In this folder, the following files are located:
			\begin{enumerate}
				\item a\_Dedication.tex
				\begin{itemize}
					\item you can dedicate your thesis to someone
				\end{itemize}
				\item b\_Publications.tex
				\begin{itemize}
					\item states all publications that were written in preparation of the dissertation
				\end{itemize}
			\end{enumerate} 				
		\subsection{03\_Content}
			In this folder, the following files are located:
			\begin{enumerate}
				\item 0\_Text.tex
				\begin{itemize}
					\item loads all single chapters
				\end{itemize}
				\item X\_iii.tex 
				\begin{itemize}
					\item contains the single chapters
					\item can be split further if deemed necessary
				\end{itemize}
			\end{enumerate} 
		\subsection{04\_Literature}\label{subsec:literature}
			In this folder, the following files are located:
			\begin{enumerate}
				\item Bibliography.bib
				\begin{itemize}
					\item contains the literature
					\item uses biblatex for the generation of the bibliography
					\item to make this work, biblatex (instead of bibtex) must be used
					\item in texmaker, change the command in the preferences for bib(la)tex to (hot key standard is still F11)
					\begin{verbatim}
						"C:/path_to/biber.exe" %
					\end{verbatim}
					\item for other LaTeX editors, please see its respective documentation to find out how to use biblatex there
					\item if your write your thesis on a Mac or a Linux distribution, the procedure is propably different. If you find out the right instructions, please forward us a brief summary (\href{mailto:c.hoffmann@tu-berlin.de}{c.hoffmann@tu-berlin.de})
				\end{itemize}
			\end{enumerate}
		\subsection{05\_Appendix}
			In this folder, the following files are located:
			\begin{enumerate}
				\item 0\_Appendix.tex
				\begin{itemize}
					\item inclues all single appendix chapters 
				\end{itemize}
				\item X\_Appendix.tex 
				\begin{itemize}
					\item contains the Xth appendix
				\end{itemize}
				\item a\_CodeLanguageSpecifications
				\begin{itemize}
					\item can be used to define a set of keywords and comment commands for a certain programming language
					\item if you want to apprehend your code in your thesis, you can either copy the code to a lstlistings environment (see example in Appendix/1\_Appendix.tex) or even include your files
					\item more information can be found in the lstlistings documentation 
				\end{itemize}
			\end{enumerate} 
		\subsection{Makeindex}\label{subsec:makeindex}
			\begin{itemize}
				\item to compile the file correctly, makeindex and biblatex are used
				\item command line for makeindex (integrated in texmaker, hot key standard: F12): 
				\begin{verbatim}
					"C:/path_to/makeindex.exe" %.nlo -s mynomentbl.ist -o %.nls
				\end{verbatim}
				\item the necessary .ist file is located in the folder
				\item check your tex editor documentation to find out how to run makeindex in another software
				\item if your write your thesis on a Mac or a Linux distribution, the procedure is propably different. If you find out the right instructions, please forward us a brief summary (\href{mailto:c.hoffmann@tu-berlin.de}{c.hoffmann@tu-berlin.de})
			\end{itemize}
%------------------------------
% SEC: GENERAL INFORMATION
%------------------------------			
	\section{General Information}
		\begin{enumerate}
			\item the current \glqq Prüfungsordnung\grqq{} overrides the following rules if they contradict the \glqq Prüfungsordnung\grqq . 
			\item a Master's thesis is a scientific-technical documentation that must satisfy requirements regarding structure and form. It should be precisely formulated and well-written, i.e.~no ortographic or grammar mistakes, etc.
			\item the thesis should be logically strucured.
			\item the thesis should present its scientific-technical content while remaining comprehensible. Hence, the author should repeatedly put him- or herself into the position of the reader and evaluate the thesis in this regard.
			\item the Figure, i.e.~picture, diagram, photo, is preferred to long explanations.
			\item results must be tractable. Hence, the applied methods, assumptions, boundary conditions, experiments, and computer codes must be pointed out and explained in sufficient detail.
			\item calculations should be documented. This is of course difficult for large models. In this case, the code should be attached to the printed or digital appendix.
			\item the thesis should focus on the central themes and aspects. Other information should be referenced appropriately, but does not have to be repeated extensively.
		\end{enumerate}
%------------------------------
% SEC: FORM
%------------------------------		
	\section{Form}
		\begin{enumerate}
			\item the thesis must be written in German or English.
			\item sentences should be comprehensible. Germans tend to formulate complex phrases with many sub-clauses. This should be avoided.
			\item physical units must always be given and stated in SI units. Units must not be stated in brackets:
			\begin{itemize}
				\item \emph{WRONG:} Pressure $P$ [Pa]
				\item \emph{RIGHT:} Pressure $P$ in Pa
			\end{itemize}
			\item a list of symbols and a list of abbreviations must be included. This is done with the nomentbl and the acronym package in this template. Furthermore, symbols should be explained in the text after their first appearence.
			\item Figures, Tables, and Equations must be numerated and must be referenced in the text. This is automatically done using the caption package (see section \ref{sec:templates}). For example, a Figure is named Figure chapter.Num (Fig. 2.1). The numeration is done automatically in this template. In addition, Figures and Tables must also be discussed in the text.
			\item Figures, which are included in the text, should be chosen to support readabiliy and comprehension. The most important details and relevant labels especially must be readable.
			\item extensive Tables or Figures that are repeatedly referenced in the text should be put in the appendix.
			\item information or data not generated by the author must always be referenced, see information on citations below.
		\end{enumerate}
%------------------------------
% SEC: FORMAT
%------------------------------		
	\section{Format}
		The format is fixed by this document. Of course, loading additional packages is possible, but aspects, such as font or font size, have to remain unchanged.

		Although we do not recommend it, writing the thesis in Microsoft Word or others is possible. In this case, the design of this template should be imitated as close as possible.
	\section{Appearence}
		\begin{enumerate}
			\item the format of the page numbering is already specified and may not be changed.
			\item there is a maximum of four indenture levels in the text and a maximum of three levels in the table of contents.
			\item the sections of the appendix are numerated alphabetically in capital Latin letters. This is already specified in this template.
			\item important aspects can be emphasized with \textit{italics}, \textbf{bold writing}, or using the \emph{emphasize command}. Underlining words should be avoided.
			\item paragraphs should not start in the last two lines of a page (\glqq Schusterjunge\grqq) or end in the first two lines of a page (\glqq Hurenkind\grqq). This is done using the nowidow package.
		\end{enumerate}
%------------------------------
% SEC: REFERENCES AND BIBLIOGRAPHY
%------------------------------		
	\section{References and Bibliography}
		Citations/references are used to
		\begin{itemize}
			\item document and justify one's own statements,
			\item differ between one's one statements and those made by others,
			\item help the reader to assess the origin of a statement
		\end{itemize}
		All information not generated by the author must be marked with a short reference, which is accompanied by the extensive reference in the bibliography. It is not important if this information appers directly or indirectly in the text. We use either the authoryear or the numerical short citation. The most important rule is: The references must be complete and follow a consistent format. This is more important than following a specific citation style. If possible, the doi/ISBN of an article/book should be part of the citation. This is also included automatically in this template.

		In the following, a few examples for the authoryear short reference are stated. For more information, the reader is referred to the documenation of the biblatex package, which is placed in the documentation folder of this document. For citations, biblatex is used because it is compatible with UTF-8. Hence, Umlaute, such as ä, do not have to be rewritten as was the case in bibtex. Example: \parencite{Muller2018}, see .bib file.
		\begin{itemize}
			\item \parencite{Coker2007}, \textcite{Coker2007}
			\item \parencite{Abrams1975}, \textcite{Abrams1975}
			\item \parencite{benzene_nist2017}, \textcite{benzene_nist2017}
			\item \parencite{Cuda2012}, \textcite{Cuda2012}
		\end{itemize}
		These commands are used as
		\begin{itemize}
			\item \textcite{Abrams1975} stated that thermodynamics are great.
			\item Thermodynamics are great \parencite{Abrams1975}.
			\end{itemize}
%------------------------------
% SEC: LANGUAGE
%------------------------------			
	\section{Language: English or German?}\label{sec:language}
		The language of this document is set with the babel package. The order of the loaded languages determines the default language. Usually, ngerman is default (and hence the \emph{second}) language. The babel package automatically sets the right names for Tables and Figures and provides the correct hyphenation. In case the thesis is written in English, the order of the languages when loading the babel package must be changed.

		The language of the list of symbols is chosen via the option german. This also affects the entry in the table of contents. If the thesis is written in English, the option german can be removed.

		The list of abbreviations must be changed in the respective file f\_abbrevi\-ations.tex as there is no package option available. The entry in the table of contents is changed in the same file.

		If the language is English, it is recommended to change the output decimal marker  for SI units (siunitx package) to a dot. This can be changed in the a\_Packages.tex file.
%------------------------------
% SEC: TEMPLATES
%------------------------------		
	\section{Templates}\label{sec:templates}
		\subsection{Units}
			Units are very important. However, there are some rules when typesetting units. First of all, they are never written in italics. They should also have the right space between them. For this purpose, the siunitx package is suggested.\\
Units can be written as $R=\SI{8.314}{\joule\per\mole\per\kelvin}$. 
		\subsection{Figures}
			\begin{figure}[tbh]
				\centering
				\includegraphics[scale=1]{dbta_logo}
				\caption[This is the caption of the figure in the List of Figures]{This is the caption of the Figure in the text. Is is placed \emph{below} the Figure. It can be longer here and contain additional information, such as references or keys for the graphs. Note that one line captions are justified. A full stop is automatically added after the last sign}
			\end{figure}
		\subsection{Tables}
			\begin{table}[tbh]
				\centering
				\caption[This is the caption of the Table in the List of Tables]{This is the caption of the Table in the text. Is is placed \emph{above} the table. It can be longer and contain additional information. Vertical lines should be avoided in tables. A full stop is automatically added after the last sign}
				\begin{tabular}{c c c}
					\toprule
					Entry 1 & Entry 2 & Entry 3 \\
					Unit 1 & Unit 2 & Unit 3 \\
					\midrule
					1 & 2 & 3 \\
					4 & 5 & 6 \\
					\bottomrule
				\end{tabular}
			\end{table}
			\cleardoublepage
%------------------------------
% SEC: SECTION HEADER EXAMPLE
%------------------------------			
	\section[head={short version of chapter heading}, tocentry={The Sectioning Command for Chapters Supports not only the Heading Text Itself but also a Short Version Whose Use can be Controlled}]{The Sectioning\\
Command for Chapters\\
Supports not only\\
the Heading Text Itself\\
but also\\
a Short Version Whose\\
Use can be Controlled}
		If your chapters or sections have long titles (which they should not have), you can use an optional argument for chapter or section commands to shorten it in the header. You can even manipulate the chapter or section title. Normally, the use of this option is not necessary.