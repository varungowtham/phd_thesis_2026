\chapter{Guidelines}
\section{General Information}
\begin{enumerate}
\item The current \glqq Prüfungsordnung\grqq{} overrides the following rules if they contradict the \glqq Prüfungsordnung\grqq . 
\item A Master's thesis is a scientific-technical documentation that must satisfy requirements regarding structure and form. It should be precisely formulated and well-written, i.e.~no ortographic or grammar mistakes, etc.
\item The thesis should be logically strucured.
\item The thesis should present its scientific-technical content while remaining comprehensible. Hence, the author should repeatedly put himself into the position of the reader and question his thesis in this regard.
\item The Figure, i.e.~picture, diagram, photo, is preferred to long explanations.
\item Results must be tractable. Hence, the applied methods, assumptions, boundary conditions, experiments, and computer codes must be pointed out and explained in sufficient detail.
\item Calculations should be documented. This is of course difficult for large models. In this case, the code should be attached to the printed or digital appendix.
\item The thesis should focus on the central themes and aspects. Other information should be referenced appropriately, but does not have to be repeated extensively.
\end{enumerate}
\section{Form}
\begin{enumerate}
\item The thesis must be written in German or English.
\item Sentences should be comprehensible. Germans tend to formulate complex phrases with many sub-clauses. This should be avoided.
\item Physical units must always be given and stated in SI units. Units must not be stated in brackets:
\begin{itemize}[label={-}]
\item \emph{WRONG:} Pressure $P$ [Pa]
\item \emph{RIGHT:} Pressure $P$ in Pa
\end{itemize}
\item A list of symbols and a list of abbreviations must be included. This is done with the nomentbl and the acronym package in this template. Furthermore, symbols should be explained in the text after their first appearence.
\item Figures, Tables, and Equations must be numerated and must be referenced in the text. This is automatically done using the caption package (see section \ref{sec:templates}). For example, a Figure is named Figure chapter.Num (Fig. 2.1). The numeration is done automatically in this template. In addition, Figures and Tables must also be discussed in the text.
\item Figures, which are included in the text, should be chosen to support readabiliy and comprehension. The most important details and relevant labels especially must be readable.
\item Extensive Tables or Figures that are repeatedly referenced in the text should be put in the appendix.
\item Information or data not generated by the author must always be referenced, see information on citations below.
\end{enumerate}
\section{Format}
The format is fixed by this document. Of course, loading additional packages is possible, but aspects, such as font or font size, have to remain unchanged.

Although we do not recommend it, writing the thesis in Microsoft Word or others is possible. In this case, the design of this template should be imitated as close as possible.
\section{Appearence}
\begin{enumerate}
\item The format of the page numbering is already specified and may not be changed.
\item There is a maximum of four indenture levels in the text and a maximum of three levels in the table of contents.
\item The sections of the appendix are numerated alphabetically in capital Latin letters. This is already specified in this template.
\item Important aspects can be emphasized with \textit{italics}, \textbf{bold writing}, or using the \emph{emphasize command}. Underlining words should be avoided.
\item Paragraphs should not start in the last two lines of a page (\glqq Schusterjunge\grqq) or end in the first two lines of a page (\glqq Hurenkind\grqq). This is done using the nowido package.
\end{enumerate}
\section{References and Bibliography}
Citations/references are used to
\begin{itemize}[label={-}]
\item document and justify one's own statements,
\item differ between one's one statements and those made by others,
\item help the reader to assess the origin of a statement
\end{itemize}
All information not generated by the author must be marked with a short reference, which is accompanied by the extensive reference in the bibliography. It is not important if this information appers directly or indirectly in the text. We use either the authoryear or the numerical short citation. The most important rule is: The references must be complete and follow a consistent format. This is more important than following a specific citation style. If possible, the doi/ISBN of an article/book should be part of the citation. This is also included automatically in this template.

In the following, a few examples for the authoryear short reference are stated. For more information, the reader is referred to the documenation of the biblatex package, which is placed in the documentation folder of this document. For citations, biblatex is used because it is compatible with UTF-8. Hence, Umlaute, such as ä, do not have to be rewritten as was the case in bibtex. Example: \parencite{Muller2018}, see .bib file.
\begin{itemize}[label={-}]
\item \parencite{Coker2007}, \textcite{Coker2007}
\item \parencite{Abrams1975}, \textcite{Abrams1975}
\item \parencite{benzene_nist2017}, \textcite{benzene_nist2017}
\item \parencite{UNIFAC2017}, \textcite{UNIFAC2017}
\item \parencite{Cuda2012}, \textcite{Cuda2012}
\end{itemize}
These commands are used as
\begin{itemize}[label={-}]
\item \textcite{Abrams1975} stated that thermodynamics are great.
\item Thermodynamics are great \parencite{Abrams1975}.
\end{itemize}

\section{Language: English or German?}
The language of this document is set with the babel package. The order of the loaded languages determines the default language. Usually, ngerman is default (and hence the \emph{second}) language. The babel package automatically sets the right names for Tables and Figures and provides the correct hyphenation. In case the thesis is written in English, the order of the languages when loading the babel package must be changed.

The language of the list of symbols is chosen via the option german. This also affects the entry in the table of contents. If the thesis is written in English, the option german can be removed.

The list of abbreviations must be changed in the respective file f\_abbrevi\-ations.tex as there is no package option available. The entry in the table of contents is changed in the same file.
\section{Templates}\label{sec:templates}
\subsection{Units}
Units are very important. However, there are some rules when typesetting units. First of all, they are never written in italics. They should also have the right space between them. For this purpose, the siunitx package is suggested.\\
Units can be written as $R=\SI{8.314}{\joule\per\mole\per\kelvin}$.
\subsection{Figures}
\begin{figure}[tbh]
\centering
\includegraphics[scale=1]{dbta_logo}
\caption[This is the caption of the figure in the List of Figures]{This is the caption of the Figure in the text. Is is placed \emph{below} the Figure. It can be longer here and contain additional information, such as references or keys for the graphs. Note that one line captions are justified. A full stop is automatically added after the last sign}
\end{figure}
\subsection{Tables}
\begin{table}[tbh]
\centering
\caption[This is the caption of the Table in the List of Tables]{This is the caption of the Table in the text. Is is placed \emph{above} the table. It can be longer and contain additional information. Vertical lines should be avoided in tables. A full stop is automatically added after the last sign}
\begin{tabular}{c c c}
\toprule
Entry 1 & Entry 2 & Entry 3 \\
Unit 1 & Unit 2 & Unit 3 \\
\midrule
1 & 2 & 3 \\
4 & 5 & 6 \\
\bottomrule
\end{tabular}
\end{table}
\cleardoublepage
\section[head={short version of chapter heading}, tocentry={The Sectioning Command for Chapters Supports not only the Heading Text Itself but also a Short Version Whose Use can be Controlled}]{The Sectioning\\
Command for Chapters\\
Supports not only\\
the Heading Text Itself\\
but also\\
a Short Version Whose\\
Use can be Controlled}
If your chapters or sections have long titles (which they should not have), you can use an optional argument for chapter or section commands to shorten it in the header. You can even manipulate the chapter or section title. Normally, the use of this option is not necessary.