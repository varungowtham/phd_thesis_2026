%------------------------------
% DOCUMENT AND ENCODING
%------------------------------
\usepackage[
	english,				% included language
	ngerman					% DEFAULT language
]{babel} 					% language package
\usepackage[
	nodayofweek				% do not print day of week with \today command
]{datetime}					% package for dates
\usepackage[
	T1						% font encoding in western european style
]{fontenc}					% font encoding
\usepackage{ifthen}			% allows for if-then-else commands
\usepackage[
	utf8					% UTF-8 
]{inputenc}					% input encoding (the .tex files)
\usepackage[
    automark, 				% mark chapter and section names
    ilines 					% separation line ragged left ausrichten
]{scrlayer-scrpage}

%------------------------------
% Tables
%------------------------------
\usepackage{booktabs}		% special table properties
\usepackage{longtable} 		% tables over several pages
\usepackage{multirow}		% columns in several rows
\usepackage{tabularx} 		% better tabulars

%------------------------------
% GRAPHICS
%------------------------------
\usepackage{graphicx}		% for including pictures
\usepackage{overpic}		% overlay to add text to figures

%------------------------------
% FONT, MATH, AND SYMBOLS
%------------------------------
\usepackage{amsmath} 		% mathematical environments and commands
\usepackage{amssymb}		% mathematical symbols
\usepackage[
	makeroom				% extend equation appropriately
]{cancel}  					% for elimating arrows in equations
\usepackage{chemfig}		% package for drawing molecules
\usepackage{courier}		% courier font for lstlistings
\usepackage{empheq}			% frames around equations
\usepackage{icomma}			% distance after comma in math environment
\usepackage[
	osf,					% old-style figures
	sc						% with small caps	
]{mathpazo} 				% contains the Palatino font
\usepackage{mathtools}		% some additional symbols
\usepackage[
version=4					% use version X of the package
]{mhchem}					% chemical formulas
\usepackage{nicefrac}		% makes fractions in lines nicer
\usepackage[
	output-decimal-marker={,}, % decimal marker (german: , and english: .)
	range-phrase={--},		% for units 4 -- 10 K
	range-units=single,		% it is 4 -- 10 K and not 4K - 10K
	retain-explicit-plus,	% \SI{1E+4}{} = 1E+04 in the text
	output-exponent-marker=\text{E}, % marker for \SI{1e-4}{} = 1E-04 in the text
]{siunitx}					% formating of SI units

%------------------------------
% TEXT
%------------------------------
\usepackage{blindtext}		% possibility to easily add blindtext
\usepackage[
	font={small,it},		% write caption small and italic
	labelfont=bf,			% label font bold
	textformat=period,		% caption ends automatically with full stop
]{caption}					% caption package
\usepackage{enumitem}		% modification of lists
\usepackage{footnote} 		% footnotes in floats
\usepackage[
	framemethod=TikZ		% use TikZ for frames
]{mdframed} 				% frames, e.g. for environments
\usepackage[
	all						% sets widow and orphan penalties for the whole document
]{nowidow}					% widows and orphans
\usepackage{placeins}		% is used to avoid floating objects after the \FloatBarrier command
\usepackage[
	onehalfspacing			% 1.5 distance between lines
]{setspace} 				% package for line spacing
\usepackage{subcaption}		% subfigures
\usepackage{xcolor}			% colors are added

%------------------------------
% References
%------------------------------
\usepackage[
	backref=true,			% backref to page with citation in references
	bibencoding=utf8,		% .bib file is utf-8 encoded
	citestyle=authoryear,	% citation stlye in text
	defernumbers=true,		% correct numbering for numerical references with multiple lists of references
	doi=true,				% print doi in references
	giveninits=true,		% abbreviate first names to initials
	hyperref=true,			% use hyperref
	isbn=true,				% print isbn in references
	maxbibnames=99,			% maximum number of authors (to avoid et al./u.a. in references
	sortlocale=auto,		% sorting of the references automatically using babel
	style=authoryear,		% style of the references
	uniquename=init,		% see biblatex documentation (basically checks if all authors can be identified by their last name)
	url=true				% print url in the references
]{biblatex}
\usepackage[
	babel=true,				% check language with babel
	english=american,		% for english: american quotation marks
	german=guillemets,		% for german: german quotation marks
	strict=true				% all warning are turned into errors
]{csquotes}					% quotation marks etc. in references

%------------------------------
% LISTS OF SYMBOLS AND ABBREVIATIONS
%------------------------------
\usepackage{acronym}		% package for list of abbreviations
\usepackage[
	german,					% german is used as language
	intoc					% list of symbols appears in list of content
]{nomentbl}					% package for list of symbols

%------------------------------
% CODE
%------------------------------
\usepackage{listings}		% for code publication, e.g. in the appendix
\usepackage{scrhack}		% package to achieve full compability between KOMA and lstlistings

%------------------------------
% HYPERREF
%------------------------------
%\usepackage{hyperref}		% is loaded by the pdfx package automatically
\usepackage[a-1b]{pdfx}		% package for PDF/A-1b conformity, if this is not necessary: remove the pdfx package and simply use hyperref