\newcommand{\nomgrouphead}[1]{\sffamily\Large\bfseries #1 \hfill} % writes group names, i.e. greek symbols, in bold 

\newcommand{\nomgroupX}[5]{								% writes group head for each group (see below). If you have issues with groups ending at unfortunate places, you can play aroung with the 6ex value
	\nomgrouphead{#1} \endgroup  \\ \nopagebreak \\ \bfseries #2 & \bfseries #3 & \bfseries #4 & \bfseries #5 \\ \bottomrule \\[-6ex] \nopagebreak \begingroup
}

\makenomenclature
%------------------------------
% ENGLISH
%------------------------------
\iflanguage{english}{
	\renewcommand\nomgroup[1]{%				& defines all different groups, can be extended if necessary
  		\ifx#1L\relax	% if the optional argument is L
    	\nomgroupX{Latin Symbols}{Symbol}{Description}{}{Unit}
  		\fi
  		\ifx#1G\relax % if the optional argument is G
    	\nomgroupX{Greek Symbols}{Symbol}{Description}{}{Unit}    
  		\fi
  		\ifx#1X\relax
    	\nomgroupX{Superscripts}{Symbol}{Description}{}{}       
  		\fi
  		\ifx#1Z\relax
   		\nomgroupX{Subscripts}{Symbol}{Description}{}{}    
  		\fi
  		\ifx#1C\relax
		\nomgroupX{Constants}{Symbol}{Description}{}{Unit}    
  		\fi
  		\ifx#1D\relax
		\nomgroupX{Dimensionless Numbers}{Symbol}{Description}{}{Definition}       
  		\fi
  		\ifx#1O\relax
		\nomgroupX{Operators}{Symbol}{Description}{}{Definition}      
  		\fi
  		\ifx#1I\relax
		\nomgroupX{Indices}{Symbol}{Description}{}{}     
  		\fi          
	}
%------------------------------
% DEUTSCH
%------------------------------	
}{\iflanguage{ngerman}{
	\renewcommand\nomgroup[1]{%								& defines all different groups, can be extended if necessary
  		\ifx#1L\relax	% if the optional argument is L
    	\nomgroupX{Lateinische Symbole}{Symbol}{Beschreibung}{}{Einheit}
  		\fi
  		\ifx#1G\relax % if the optional argument is G
    	\nomgroupX{Griechische Symbole}{Symbol}{Beschreibung}{}{Einheit}  
  		\fi
  		\ifx#1X\relax
    	\nomgroupX{Superskripte}{Symbol}{Beschreibung}{}{}       
  		\fi
  		\ifx#1Z\relax
    	\nomgroupX{Subskripte}{Symbol}{Beschreibung}{}{}    
  		\fi
  		\ifx#1C\relax
		\nomgroupX{Konstanten}{Symbol}{Beschreibung}{}{Einheit}     
  		\fi
  		\ifx#1D\relax
		\nomgroupX{Dimensionslose Kennzahlen}{Symbol}{Beschreibung}{}{Definition}      
  		\fi
  		\ifx#1O\relax
		\nomgroupX{Operatoren}{Symbol}{Beschreibung}{}{Definition}     
 		\fi
  		\ifx#1I\relax
		\nomgroupX{Indizes}{Symbol}{Beschreibung}{}{}      
  		\fi          
		}
	}{}
}