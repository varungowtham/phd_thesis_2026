\newcommand{\nomgrouphead}[1]{\Large\bfseries #1 \hfill} % writes group names, i.e. greek symbols, in bold 

\newcommand{\nomgroupX}[2]{								% writes group head for each group (see below). If you have issues with groups ending at unfortunate places, you can play aroung with the 6ex value
	\nomgrouphead{#1} \endgroup  \\ \nopagebreak \\ #2 \\ \bottomrule \\[-6ex] \nopagebreak 			\begingroup
}

\makenomenclature
%------------------------------
% ENGLISH
%------------------------------
\iflanguage{english}{
	\renewcommand\nomgroup[1]{%				& defines all different groups, can be extended if necessary
  		\ifx#1L\relax	% if the optional argument is L
    		\nomgroupX{Latin Symbols}{\bfseries Symbol&\bfseries Description&&\bfseries Unit}
  		\fi
  		\ifx#1G\relax % if the optional argument is G
    	\nomgroupX{Greek Symbols}{\bfseries Symbol&\bfseries Description&&\bfseries Unit}    
  		\fi
  		\ifx#1X\relax
    	\nomgroupX{Superscripts}{\bfseries Symbol&\bfseries Description&&}       
  		\fi
  		\ifx#1Z\relax
   		\nomgroupX{Subscripts}{\bfseries Symbol&\bfseries Description&&}     
  		\fi
  		\ifx#1K\relax
		\nomgroupX{Constants}{\bfseries Symbol&\bfseries Description&&\bfseries Unit}     
  		\fi
  		\ifx#1N\relax
		\nomgroupX{Dimensionless Numbers}{\bfseries Symbol&\bfseries Description&&\bfseries Definition}       
  		\fi
  		\ifx#1O\relax
		\nomgroupX{Operators}{\bfseries Symbol&\bfseries Description&&\bfseries Definition}      
  		\fi
  		\ifx#1I\relax
		\nomgroupX{Indices}{\bfseries Symbol&\bfseries Description&&}      
  		\fi          
	}
%------------------------------
% DEUTSCH
%------------------------------	
}{\iflanguage{ngerman}{
	\renewcommand\nomgroup[1]{%								& defines all different groups, can be extended if necessary
  		\ifx#1L\relax	% if the optional argument is L
    	\nomgroupX{Lateinische Symbole}{\bfseries Symbol&\bfseries Beschreibung&&\bfseries Einheit}
  		\fi
  		\ifx#1G\relax % if the optional argument is G
    	\nomgroupX{Griechische Symbole}{\bfseries Symbol&\bfseries Beschreibung&&\bfseries Einheit}    
  		\fi
  		\ifx#1X\relax
    	\nomgroupX{Superskripte}{\bfseries Symbol&\bfseries Beschreibung&&}       
  		\fi
  		\ifx#1Z\relax
    	\nomgroupX{Subskripte}{\bfseries Symbol&\bfseries Beschreibung&&}     
  		\fi
  		\ifx#1K\relax
		\nomgroupX{Konstanten}{\bfseries Symbol&\bfseries Beschreibung&&\bfseries Einheit}     
  		\fi
  		\ifx#1N\relax
		\nomgroupX{Dimensionslose Kennzahlen}{\bfseries Symbol&				\bfseries Beschreibung&&\bfseries Definition}       
  		\fi
  		\ifx#1O\relax
		\nomgroupX{Operatoren}{\bfseries Symbol&\bfseries Beschreibung&&\bfseries Definition}      
 		\fi
  		\ifx#1I\relax
		\nomgroupX{Indizes}{\bfseries Symbol&\bfseries Beschreibung&&}      
  		\fi          
		}
	}{}
}