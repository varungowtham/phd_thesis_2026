
\addchap{Kurzfassung/Abstract}
\label{cha:abstract}

Die wissenschaftliche Abschlussarbeit -- sei es Bachelor-, Master- oder Doktorarbeit -- soll einen wissenschaftlich-technischen Inhalt verständlich vermitteln und gleichzeitig allgemeinen Anforderungen bezüglich Struktur und Form entsprechen.
Die hier vorliegende \enquote{Arbeit} mit ihren Richtlinien und Hinweisen soll den Kandidaten des Fachgebiets DBTA verdeutlichen, wie eine Abschlussarbeit strukturiert und formatiert werden kann, um allen Anforderungen von Prüfungsamt und Korrektoren zu entsprechen.
Das zugrundeliegende LaTeX-Template sorgt für eine ansprechende Formatierung und ermöglicht es dem Verfasser einer Abschlussarbeit, sich voll auf den Inhalt und die sorgfältige Formulierung desselben zu konzentrieren.

Die Kurzfassung, im Englischen \enquote{Abstract} genannt, fasst die gesamte Arbeit überblicksartig zusammen und soll den Leser zum Weiterlesen motivieren und ihm bei der Entscheidung helfen, ob die vorliegende Arbeit für ihn relevant ist oder nicht.
Die Kurzfassung wird in der Regel ganz am Ende des Schreibprozesses geschrieben, wenn der gesamte inhaltliche Text der Arbeit fertig ist. Sie sollte auch für den Leser verständlich sein, der nicht die ganze Arbeit gelesen hat. Unabhängig von dieser Kurzfassung kann es nach dem Diskussionskapitel des Hauptteils, vor den Anhängen, noch ein  Kapitel \enquote{Zusammenfassung} geben, welches für die Leser gedacht ist, die die Arbeit gelesen haben und denen noch einmal die interessantesten Ergebnisse genannt werden sollen.


