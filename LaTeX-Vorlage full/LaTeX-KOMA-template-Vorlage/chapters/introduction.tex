%% example text content
%% scrartcl and scrreprt starts with section, subsection, subsubsection, ...
%% scrbook starts with part (optional), chapter, section, ...
\chapter{Einleitung/Introduction}

Die in diesem Dokument aufgeführten Richtlinien sollen eine einigermaßen einheitliche Form der am Fachgebiet \enquote{Dynamik und Betrieb technischer Anlagen} (DBTA) angefertigten Abschlussarbeiten gewährleisten.
Sie sind so gewählt, dass die äußere Form den Vorgaben des Prüfungsamts folgt. Zusätzlich wurde ein großer Wert auf eine typographisch ansprechende Formatierung und Seitengestaltung gelegt.

Nur sehr wenige Korrektoren fühlen sich von Rand- und Absatzlosen \enquote{Textwüsten} angezogen. Um einen solchen Eindruck zu vermeiden, werden relativ strenge Vorgaben für Ränder und Zeichen pro Seite gemacht. Auf den ersten Blick mag dies ungewohnt erscheinen, die Werte wurden aber bewusst so gewählt, um eine Ermüdung des Lesers möglichst zu vermeiden. Eine zu große Anzahl an Zeichen pro Zeile bzw. Seite würde auf den Leser \enquote{demotivierend} wirken. Da der Leser von Abschlussarbeiten zunächst oft auch deren Korrektor oder Gutachter ist, wollen wir das zum Besten des Verfassers der Abschlussarbeit ausdrücklich verhindern.

Die Richtlinien sind eindeutig als Hilfestellung zu sehen. Mit diesem Dokument wird außerdem ein Beispiel zur Verfügung gestellt, welches ein Gefühl für den angestrebten Gesamteindruck der anzufertigenden Arbeiten vermittelt.
Bei Nutzung des LaTeX-Codes, der für dieses Dokument verwendet wurde, werden automatisch alle Richtlinien eingehalten und wer auch immer die nächste Abschlussarbeit am Fachgebiet DBTA schreibt, kann sich voll auf den Inhalt konzentrieren, da das typographische \enquote{drumherum} für die gute Präsentation sorgt.\\

Für den Aufbau der Abschlussarbeit sei folgende Struktur gewählt:
\begin{itemize}
	\item Titelseite, die gleichzeitig als \enquote{Schmutztitel} dient
	\item Eidesstattliche Versicherung
	\item (optional) Aufgabenstellung (nur einfügen, falls eine Prüfungsordnung es verlangt)
	\item (optional) Colophon mit Hinweisen zu technischer Herstellung des Dokuments
	\item (optional) Widmung (dedication)
	\item (optional) Danksagung
	\item (optional) Geleitwort (preface) mit Hinweisen zu Vorgeschichte und Entstehungsgeschichte der Arbeit
	\item Inhaltsverzeichnis, welches so früh wie möglich stehen sollte, um dem Leser einen Gesamtüberblick zu vermitteln
	\item Kurzfassung (abstract)
	\item Abbildungsverzeichnis (obligatorisch, sobald es mehrere Abbildungen in der Arbeit gibt)
	\item Tabellenverzeichnis (obligatorisch, sobald es mehrere Tabellen in der Arbeit gibt)
	\item Glossar / Abkürzungsverzeichnis / Liste der Symbole
	\item Hauptteil inklusiver aller Unterkapitel, Beginn der Seitenzahlen mit arabischen Ziffern
	\begin{itemize}
		\item Einleitung, Problemstellung
		\item Theorie, Begriffsklärungen, Stand des Wissens
		\item \dots
		\item Ergebnisse
		\item Diskussion
		\item Schlussfolgerungen
		\item Zusammenfassung
		
	\end{itemize}
	\item (optional) Anhänge inklusive Quellcode, Beweisen, Algorithmen, vertiefenden Beispielen, Datentabellen, die im Hauptteil den Lesefluss unnötig unterbrechen würden
	\item Literaturverzeichnis (bibliography)
\end{itemize}

In den folgenden Kapiteln dieses Dokuments werden die Richtlinien für Abschlussarbeiten aufgelistet (Kapitel 2), für Abschlussarbeiten relevante Normen genannt (Kapitel 3) und eine Anleitung gegeben, wie man eine Abschlussarbeit sinnvollerweise mit LaTeX erstellt (Kapitel 4). Kapitel 5 ist lediglich ein Platzhalter zur Demonstration von Unterkapiteln und der Einbindung von Graphiken. Im Anhang werden hier lediglich im Kapitel \enquote{Quellcode} zwei kleine Java-Dateien gelistet. Zuletzt folgt das Literaturverzeichnis im Name-Jahr-Stil mit Rückreferenzen zu den Vorkommen im Text.\\

Now you are able to write your own document. Always keep in mind: it's
the \emph{content} that matters, not the form. But good typography is
able to deliver the content much better than information set with bad
typography. This template allows you to concentrate on writing good
content while the form is done by the template definitions.


