%% example text content
%% scrartcl and scrreprt starts with section, subsection, subsubsection, ...
%% scrbook starts with part (optional), chapter, section, ...
\chapter{Abschlussarbeit mit LaTeX erstellen}
\section{TeX-Distributionen}

Empfohlene Tex-Distributionen:
\begin{itemize}
\item TeX Live, 
\item MiKTeX (Windows) 
\end{itemize}


\subsection{TeX Live}
TeX Live ist die wohl vollständigste Tex-Distribution und ist auf allen Plattformen verfügbar.
\subsection{MiKTeX}
MiKTeX ist eine TeX-Distribution, die auf Windows-Betriebssystem verfügbar ist. Der Download kann direkt über die Homepage erfolgen:

https://miktex.org/download\\

Ende Juni 2017 ist Version 2.9.6 die aktuelle Version von MiKTeX.
Basic MiKTeX kann so eingestellt werden, dass benötigte Pakete später nachgeladen werden. 
Bei der Auswahl \enquote{Complete MiKTeX} werden im Sinne einer Komplettinstallation alle Pakete auf einmal geladen und installiert.
Die Gesamtgröße des Pakets beträgt rund 2,4 GB und das Herunterladen kann dementsprechend 30-40 Minuten dauern.
Die Initialisierung der Installation hat bereits 15 bis 20 Minuten in Anspruch genommen. Die tatsächliche Installation braucht auch einige Minuten, sodass mit insgesamt mehr als 30 Minuten zu rechnen ist; also Geduld haben und nicht gleich abbrechen, auch wenn sich der Fortschrittsbalken mal nicht rührt.\\

Zusätzliche Herausforderung: Das Paket Asymptote wird \textit{nicht} mit MiKTeX mitgeliefert und muss bei Bedarf manuell installiert werden, zum Beispiel, indem es mit Hilfe eines lokalen texmf-Baums aus einem Verzeichnis gemäß  der TeX Directory Structure (TDS) eingebunden wird. Beschreibung und Anleitungen gibt es natürlich einige im Internet, hier zwei Links zum Einstieg in die Thematik:
\begin{itemize}
	\item https://tex.stackexchange.com/questions/83035/using-asymptote-with-miktex
	\item https://tex.stackexchange.com/questions/69483/create-a-local-texmf-tree-in-miktex
\end{itemize}


Bei den MiKTeX Settings (aufzurufen z.B. über das Windows-Startmenü) kann es ausreichen, die Nicht-Admin-Version zu verwenden und dort einen root-path einzugeben (+ Refresh FNDB)

\section{TeX-Entwicklungsumgebungen}

TeXstudio, Texmaker, TeXnicCenter (Windows), TeXShop (macOS)
Es ist Geschmacks- und Gewöhnungssache, mit welcher Entwicklungsumgebung man am produktivsten arbeitet, deshalb wird hier keine Empfehlung für einen bestimmten Editor ausgesprochen.

\section{KOMA-Script}
Das in der Latex-Vorlage verwendete KOMA-Script ist ausführlich in der Datei scrguide dokumentiert.

Für Abschlussarbeiten, die nur am Fachgebiet DBTA angefertigt werden, ist die in der Latex-Vorlage vorbereitete Titelseite zu verwenden. Sollten mehrere Lehrstühle oder Kooperationspartner aus der Industrie beteiligt sein, kann diese Titelseite von der Auswahl der Logos abgewandelt werden. Es kann auch komplett auf Logos verzichtet werden.


\section{Literaturverwaltung, Bibliographie, Referenzen}

Wer schon gute Erfahrung mit bibtex und natbib hat, kann dabei bleiben, in allen anderen Fällen wird empfohlen, die aktuelleren/moderneren Pakete/Programme biber und biblatex zu verwenden. Mit den Literaturverwaltungsprogrammen Citavi oder JabRef oder Mendeley lassen sich Einträge für die \enquote{.bib}-Datei erzeugen, die letztendlich von biber verarbeitet und mit biblatex im Quellcode eingebunden wird. Da biber und biblatex für eine Zusammenarbeit konzipiert sind, ist bitte darauf zu achten, dass die Versionen von biber und biblatex zueinander passen / aktuell sind.\\

Hier passende Hinweise aus der LaTeX-Community:

\begin{itemize}
	\item http://golatex.de/wichtige-hinweise-erstellung-von-literaturverzeichnissen-t11964.html
\end{itemize}

Hinweise zur Einbindung von biber in den jeweiligen Editor:

\begin{itemize}
	\item http://www.texwelt.de/wissen/fragen/1909/wie-verwende-ich-biber-in-meinem-editor
\end{itemize}

Zum Einlesen in die Unterschiede hier ein Link von einer Diskussion, die im Jahr 2011 begann:

\begin{itemize}
	\item https://tex.stackexchange.com/questions/25701/bibtex-vs-biber-and-biblatex-vs-natbib
\end{itemize}

Bei allen bib- und tex-Dateien ist darauf zu achten, dass durchgehend eine UTF-8 Kodierung verwendet wird. Bei TeXstudio und Notepad++ ist bestätigt, dass dies direkt einstellbar ist. Bei Verwendung anderer Editor ist dies entsprechend zu überprüfen.



