

\addchap{Glossar}
\label{cha:glossary}

Im Glossar werden im Sinne eines Abkürzungs-/Symbolverzeichnisses Begriffe aufgelistet und kurz erläutert, die zum Verständnis der Arbeit relevant, aber einem Fachfremden (!) nicht allgemein bekannt sind. Beispielsweise wäre es nicht sinnvoll, die Abkürzung \enquote{\zB} hier erklären zu wollen. Wenn in der Arbeit mathematische Modelle stehen, bietet es sich an, für deren Notation ein eigenes, gemeinsames Verzeichnis zu erstellen und dieses im Anhang einzufügen.

\begin{tabular}{lp{10cm}}
DAE& \enquote{Differential-Algebraic-Equationsystem}, Differential-Algebra-Gleichungssystem \\
LaTeX& \enquote{Lamport TeX}, Erweiterung des Textsatzsystems \enquote{TeX} mittels Makros  \\
MATHML& \enquote{MATHEmatical Markup Language}, XML-Format zur Darstellung mathematischer Formeln \\
NLE& \enquote{Non-Linear-Equationsystem}, Nicht-Lineares Gleichungssystem \\
ODE& \enquote{Ordinary Differential Equation(system)}, Gew"ohnliche(s) Differentialgleichung(ssystem) \\
PDE& \enquote{Partial Differential Equation(system)}, Partielle(s) Differentialgleichung(ssystem) \\
TEX& Textsatzsystem zur Erstellung l"angerer Texte und mathematischer Formeln \\
XML& \enquote{Extensible Markup Language}, \enquote{Erweiterbare Auszeichnungssprache} zur Darstellung strukturierter Daten\\
\end{tabular}


%\glsresetall %% all glossary entries should be used in long form (again)

