%%%% Time-stamp: <2013-02-25 10:31:09 vk>
%% ========================================================================
%%%% Disclaimer
%% ========================================================================
%%
%% created by
%%
%%      Karl Voit
%%
%%		2013-06, 2017-07 modified by Gregor Tolksdorf for DBTA, TU Berlin

%% ========================================================================
%%%% Basic settings
%% ========================================================================
%% (idea of using newcommands for basic documentclass settings from: Thomas Schlager)

\newcommand{\mypapersize}{A4}
%% e.g., "A4", "letter", "legal", "executive", ...
%% The size of the paper of the resulting PDF file.

\newcommand{\mylaterality}{twoside}
%% "oneside" or "twoside"
%% Either you are creating a document which is printed on both, left pages
%% and right pages (twoside) or you create a document which is printed
%% on right pages only (oneside).

\newcommand{\mydraft}{false}
%% "true" or "false"
%% Use draft mode? If true, included graphics are replaced by empty
%% rectangles (of same size) and overfull boxes (in margin space) are
%% marked with black box (-> easy to spot!)

\newcommand{\myparskip}{no}
%% e.g., "no", "full", "half", ...
%% How to separate paragraphs: indention ("no") or spacing ("half",
%% "full", ...).

\newcommand{\myBCOR}{0mm}
%% Inner binding correction. This value depends on the method which is
%% being used to bind your printed result. Some techniques do not
%% require a binding correction at all ("0mm"), other require for
%% example "5mm". Refer to KOMA script documentation for a detailed
%% explanation what a binding correction is and how to measure it.

\newcommand{\myfontsize}{11pt}
%% e.g., 10pt, 11pt, 12pt
%% The font size of the main text in pt (points).

\newcommand{\mylinespread}{1.2}
%% e.g., 1.0, 1.5, 2.0
%% Line spacing in %/100. For example 1.5 means 150% of the usual line
%% spacing. Please use with caution: 100% ("1.0") is fine because the
%% font was designed for it.

\newcommand{\mylanguage}{english,ngerman}
%% "english,ngerman", "ngerman,english", ...
%% NOTE: The *last* language is the active one!
%% See babel documentation for further details.

%% BibLaTeX-settings: (see biblatex reference for further description)
\newcommand{\mybiblatexstyle}{authoryear}
%% e.g., "aphabetic", "authoryear", ...
%% The biblatex style which is being used for referencing. See
%% biblatex documentation for further details and more values.

\newcommand{\mybiblatexdashed}{false}  %% "true" or "false"
%% If true: replace recurring reference authors with a dash.

\newcommand{\mybiblatexbackref}{true}  %% "true" or "false"
%% If true: create backward links from reference to citations.

\newcommand{\mybiblatexfile}{references-biblatex.bib}
%% Name of the biblatex file that holds the references.

\newcommand{\mybibliographytitle}{Literaturverzeichnis}
%% Title of the bibliography.

\newcommand{\mylistingstitle}{Quellcode-Listing}
%% Title of the list of listings.

\newcommand{\mydispositioncolor}{0,0,0}
%% e.g., "30,103,182" (blue/turquois), "0,0,0" (black), ...
%% Color of the headings and so forth in RGB (red,green,blue) values.

\newcommand{\mycolorlinks}{true}  %% "true" or "false"
%% Enables or disables colored links (hyperref package).

%\newcommand{\mytitlepage}{template/title_Thesis_TU_Berlin}
%% Your own or one of following pre-defined title pages:
%% my own template: template/title_Thesis_TU_Berlin

\newcommand{\mytodonotesoptions}{}
%% e.g., "" (empty), "disable", ...
%% Options for the todonotes-package. If "disable", all todonotes will
%% be hidden (including listoftodos).

%% Load main settings for document preamble:
\input{template/preamble}%% DO NOT REMOVE THIS LINE!

\setboolean{myaddcolophon}{true}  %% "true" or "false"
%% If set to "true": a colophon (with notes about this document
%% template, LaTeX, ...) is added after the title page.

\setboolean{myadddedication}{true}  %% "true" or "false"
%% If set to "true": a colophon (with notes about this document
%% template, LaTeX, ...) is added after the title page.

\setboolean{myaddlistoftodos}{false}  %% "true" or "false"
%% If set to "true": the current list of open todos is added after the
%% table of contents. If \mytodonotesoptions is set to "disable", no
%% list of todos is added, independent of this setting here.



%% ========================================================================
%%%% Document metadata
%% ========================================================================

%% general metadata:
\newcommand{\myauthor}{Mohammadali Khadivi}  %% also used for PDF metadata (hyperref) -> check this!
\newcommand{\mytitle}{Masterarbeit}  %% also used for PDF metadata (hyperref) -> check this!
\newcommand{\mysubject}{CO2 dehydrogenation}  %% also used for PDF metadata (hyperref) -> check this!
\newcommand{\mykeywords}{keyword1 keyword2 keywordn}  %% also used for PDF metadata (hyperref) -> check this!

%% this information is used only for generating the title page:
\newcommand{\myworktitle}{Masterarbeit}  %% official type of work like ``Master theses''
\newcommand{\mygrade}{Master of Science (M.Sc.)} %% academic title you are achieving with this work like ``Master of ...''
\newcommand{\mystudy}{Energie- und Verfahrenstechnik} %% your study like ``Arts''
\newcommand{\myuniversity}{Technische Universit\"at Berlin} %% your university/school
\newcommand{\myinstitute}{Fakult\"at III - Prozesswissenschaften\\
Institut f\"ur Prozess- und Verfahrenstechnik} %% affiliation
\newcommand{\myinstitutehead}{Univ.-Prof.\,Dipl-Ing.\,Dr.techn.~Some One} %% head of institute
\newcommand{\mysupervisor}{Prof. Dr.-Ing. Jens-Uwe Repke} %% your supervisor
\newcommand{\myevaluator}{Mein Betreuer, M.Sc.} %% your evaluator
\newcommand{\myhomestreet}{Street~42} %% your home street (with house number)
\newcommand{\myhometown}{Berlin} %% your home town
\newcommand{\myhomepostalnumber}{12345} %% your postal number of home town
\newcommand{\mysubmissionmonth}{Juli} %% month you are handing in
\newcommand{\mysubmissionyear}{2017} %% year you are handing in
\newcommand{\mysubmissiontown}{Berlin} %% town of handing in (or \myhometown)

%% additional information for generic_documentation title page
\newcommand{\myid}{120815} %% Matrikelnummer
\newcommand{\mylecture}{LECTURE} %%

\newcommand{\mydedication}{To my parents} %% dedication


%% ========================================================================
%%%% MISC command definitions
%% ========================================================================
\input{template/mycommands}

%% ========================================================================
%%%% Typographic settings
%% ========================================================================
\input{template/typographic_settings}


%% ========================================================================
%%%% MISC usepackages
%% ========================================================================

%% ... it's OK to put here your own usepackage commands ...
\usepackage{listings}	%%insert program code using the listings package



%% ========================================================================
%%%% MISC self-defined commands and settings
%% ========================================================================

%% ... it's OK to put here your own newcommand/newenvironment-definitions ...



\newcommand{\myLaT}{\LaTeX{}@TUG\xspace} %% LaTeX@TUG text "logo"

\hyphenation{ex-am-ple hy-phen-ate ver-st\"and-lich}  %% in order to use German umlauts
%% here (Ver-\"of-fent-li-chung), you have to check for
%% activated \usepackage[T1]{fontenc} in the preamble

%% override default language of babel: (be sure to know, what you're
%% doing here)
%\selectlanguage{american}
%\selectlanguage{ngerman}

%% ========================================================================
%%%% Templates
%% ========================================================================

%% template for inserting figures:
% \myfig{}%% filename
%       {}%% width/height
%       {}%% caption
%       {}%% optional (short) caption for list of figures
%       {fig:}%% label

%% acronyms in small caps: \myacro{UNESCO}



%\input{template/pdf_settings}  %% should be *last* definitions in preamble!
%% ========================================================================
%%%% begin{document}
%% ========================================================================
\begin{document}

\frontmatter                    %% KOMA: roman page numbers and such; only available in scrbook

%% Colophon must in this case be defined before titlepage is loaded

\newcommand{\mycolophon}{%%
  This document is set in Palatino.
  Insert additional comments on the tools used for writing/creation of this document here.

  The \LaTeX{} template used for this document was designed for DBTA and is based on
  \href{http://www.komascript.de/}{KOMA script}.
}

                %% defines information about editor, LaTeX, font, ...

%% Choose your desired title page:
%%%%
%% ========================================================================
%%%% Disclaimer
%% ========================================================================
%%
%% created by
%%
%%      Stefan Kroboth and Karl Voit (TU Graz)
%%
%% modified by
%%		Gregor Tolksdorf (dbta, TU Berlin)
%%

\begin{titlepage}
	\title{\mysubject}
\large  %% basic font size of the titlepage

\AddToShipoutPicture*{%
  \AtPageUpperLeft{%
    \hspace{\paperwidth}%
    \raisebox{-60mm}{%\baselineskip}{%
     \makebox[-35mm][r]{\includegraphics[width=80mm]{figures/TUBerlin_dbta_Logo}}	%%Logo Fachgebiet dbta
     \makebox[-100mm][r]{\includegraphics[width=42mm]{figures/TUBerlin_Logo_rot}}	%%Logo TU Berlin
}}}%

\begin{center}
~
\vspace{10mm}
\begin{spacing}{1}
%{\huge\bfseries\mysubject}
\huge\bfseries\mysubject
\end{spacing}
%\vspace{2mm}

{\Large  Wissenschaftliche Arbeit zur Erlangung des Grades \\ \mygrade}

\vfill

vorgelegt von

{\Large\bfseries\myauthor}
\vfill
Matrikelnummer\\
{\Large\myid}

\vfill
Unter der wissenschaftlichen Leitung von\\
{\bfseries\mysupervisor}
\vfill

%usually not mentioned on titlepage:%
Unter der wissenschaftlichen Betreuung von\\
{\bfseries\myevaluator}

\vfill

{\Large \mysubmissiontown, \mysubmissionmonth~\mysubmissionyear}
\vfill

\myuniversity\\
\myinstitute\\

\vfill


\end{center}

\end{titlepage}


  \newpage
  \large
  \thispagestyle{empty}  %% no page header or footer
  {\Large\bfseries Eidesstattliche Versicherung}
  \vspace{1mm}\\
  
Hiermit erkläre ich, dass ich die vorliegende Arbeit selbstständig und eigenhändig sowie ohne unerlaubte fremde Hilfe und ausschließlich unter Verwendung der aufgeführten Quellen und Hilfsmittel angefertigt habe.

  %%%% in case you have to say it in english, you may use the following three lines:
  %% {\Large\bfseries Declaration}
  %% \vfill
  %% I herby declare that I am the original author of this thesis and that the work presented is, to the best of my knowledge and belief, original except as acknowledged in the text.
\begin{center}
	\begin{tabular}{ccc}
		& & \\
		& & \\
		Berlin, den \today & \hspace{3cm} & \underline{\hspace{55mm}} \\
		& & \myauthor
	\end{tabular}
\end{center}
%\vfill


%% if myaddcolophon is set to "true", colophon is added:
\ifthenelse{\boolean{myaddcolophon}}{
	\newpage
	\thispagestyle{empty}  %% no page header or footer
	
	~
	\vfill
	\mycolophon
}{}

\newpage
%% if myadddedication is set to "true", dedication is added:
\ifthenelse{\boolean{myadddedication}}{
	\newpage
	\vspace*{\fill}
	\thispagestyle{empty}  %% no page header or footer
	\begin{center}
		\begin{minipage}{0.5\textwidth}
		\Large \mydedication
		\end{minipage}
	\end{center}
	\vfill
	\clearpage
}{}
\newpage
%% end of title page
            %% include title page



\addchap*{Danksagung}
\label{cha:thanks}

Ich danke Herrn A und Frau B für die Anregungen und Diskussionen zu diesem Thema. Meinen Kollegen Y und Z für die gute Zusammenarbeit während der letzten Monate und Jahre, meinen Freunden danke ich für die Rückfragen, die mir gezeigt haben, wie wichtig es immer wieder ist, den Blick über den Tellerrand und auch von außen auf das eigene Fachthema zu richten.
Letztendlich wäre ich aber ohne die Unterstützung und den Rückhalt durch meine Frau, die in der Zeit der Anfertigung dieser Arbeit leider viel zu oft ohne meine Gesellschaft auskommen musste, niemals bis zu diesem Punkt gekommen. Danke.


                %% this is a suggestion: you have to create this file on demand

\addchap*{Preface/Geleitwort/Vorwort}
\label{cha:preface}

Das Geleitwort \enquote{preface} kann der Autor der Arbeit nutzen, um etwas über die Entstehungsgeschichte der Arbeit zu schreiben.
Im Gegensatz dazu muss das \enquote{foreword} nicht unbedingt von Autor selbst stammen -- man kennt das von Büchern, die gerne mit \enquote{mit einem Vorwort von \dots} beworben werden. Dort, also im \enquote{foreword}, wird dem Leser eine Motivation gegeben, warum er die Arbeit lesen sollte.
Da in wissenschaftlichen Arbeiten die Kurzfassung bereits die Rolle der Motivation übernimmt, bietet es sich an, das Vorwort im Sinne eines Editorials/Geleitworts zu interpretieren und über die Begleitumstände der vorliegenden Arbeit zu schreiben. Wenn bereits eine Danksagung Teil der Arbeit ist, sollte man besser darauf verzichten, diese im Geleitwort zu wiederholen. Andernfalls wäre hier durchaus der richtige Ort dafür.

Diese Arbeit entstand im Sommer 2017, als Prof. Repke -- genervt von überlangen, schlecht lesbaren und mit mangelhaftem Literaturverzeichnis und fragwürdigem Zitierstil ausgestatteten Abschlussarbeiten -- mich beauftragt hat, eine Richtlinie für Abschlussarbeiten am Fachgebiet \enquote{Dynamik und Betrieb technischer Anlagen} (DBTA) zu formulieren.
Basierend auf Vorgaben des Prüfungsamts der TU-Berlin, einem Buch zum Thema Abschlussarbeiten, einer Vorlage für Dissertationen, und Richtlinien, mit denen verschiedene aktuelle Mitarbeiter schon in Kontakt kamen, habe ich eine Vorlage erstellt, mit der unter anderem genau dieses Dokument erstellt wurde.

Möge diese Vorlage noch lange von Nutzen sein, Korrektoren die Arbeit erleichtern, Studierenden und Promovierenden Struktur- und Formatierungsentscheidungen abnehmen und Lesern Vorfreude auf weitere Abschlussarbeiten aus unserem Fachgebiet bereiten.
Auch wenn das Vorwort weniger formell ist als der Haupttext, sollte man auch hier auf unangemessene Elemente wie Emoticons verzichten. So bleibt eigentlich nur noch zu hoffen, dass der Inhalt mit der Form mithalten kann ;-)\\


Berlin, Juli 2017 \hspace{6cm} \textit{Gregor Tolksdorf}



              %% this is a suggestion: you have to create this file on demand

\tableofcontents                %% this produces the table of contents - you might have guessed :-)

%% =========================================================================
%%%%%%%% Abstract (include your abstract here): %%%%%%%%%%%%%%
%% =========================================================================

\addchap{Kurzfassung/Abstract}
\label{cha:abstract}

Die wissenschaftliche Abschlussarbeit -- sei es Bachelor-, Master- oder Doktorarbeit -- soll einen wissenschaftlich-technischen Inhalt verständlich vermitteln und gleichzeitig allgemeinen Anforderungen bezüglich Struktur und Form entsprechen.
Die hier vorliegende \enquote{Arbeit} mit ihren Richtlinien und Hinweisen soll den Kandidaten des Fachgebiets DBTA verdeutlichen, wie eine Abschlussarbeit strukturiert und formatiert werden kann, um allen Anforderungen von Prüfungsamt und Korrektoren zu entsprechen.
Das zugrundeliegende LaTeX-Template sorgt für eine ansprechende Formatierung und ermöglicht es dem Verfasser einer Abschlussarbeit, sich voll auf den Inhalt und die sorgfältige Formulierung desselben zu konzentrieren.

Die Kurzfassung, im Englischen \enquote{Abstract} genannt, fasst die gesamte Arbeit überblicksartig zusammen und soll den Leser zum Weiterlesen motivieren und ihm bei der Entscheidung helfen, ob die vorliegende Arbeit für ihn relevant ist oder nicht.
Die Kurzfassung wird in der Regel ganz am Ende des Schreibprozesses geschrieben, wenn der gesamte inhaltliche Text der Arbeit fertig ist. Sie sollte auch für den Leser verständlich sein, der nicht die ganze Arbeit gelesen hat. Unabhängig von dieser Kurzfassung kann es nach dem Diskussionskapitel des Hauptteils, vor den Anhängen, noch ein  Kapitel \enquote{Zusammenfassung} geben, welches für die Leser gedacht ist, die die Arbeit gelesen haben und denen noch einmal die interessantesten Ergebnisse genannt werden sollen.


     %% include Abstract


\newpage												%% be sure a new page starts
\phantomsection \label{listoffig}		%% create a phantom page to set the anchor for the toc entry
\addcontentsline{toc}{chapter}{Abbildungsverzeichnis}	%% this is a suggestion
\listoffigures									%% create list of figures

\newpage												%% be sure a new page starts
\phantomsection \label{listoftab}		%% create a phantom page to set the anchor for the toc entry
\addcontentsline{toc}{chapter}{Tabellenverzeichnis}	%% this is a suggestion
\listoftables									%% create list of tables



\addchap{Glossar}
\label{cha:glossary}

Im Glossar werden im Sinne eines Abkürzungs-/Symbolverzeichnisses Begriffe aufgelistet und kurz erläutert, die zum Verständnis der Arbeit relevant, aber einem Fachfremden (!) nicht allgemein bekannt sind. Beispielsweise wäre es nicht sinnvoll, die Abkürzung \enquote{\zB} hier erklären zu wollen. Wenn in der Arbeit mathematische Modelle stehen, bietet es sich an, für deren Notation ein eigenes, gemeinsames Verzeichnis zu erstellen und dieses im Anhang einzufügen.

\begin{tabular}{lp{10cm}}
DAE& \enquote{Differential-Algebraic-Equationsystem}, Differential-Algebra-Gleichungssystem \\
LaTeX& \enquote{Lamport TeX}, Erweiterung des Textsatzsystems \enquote{TeX} mittels Makros  \\
MATHML& \enquote{MATHEmatical Markup Language}, XML-Format zur Darstellung mathematischer Formeln \\
NLE& \enquote{Non-Linear-Equationsystem}, Nicht-Lineares Gleichungssystem \\
ODE& \enquote{Ordinary Differential Equation(system)}, Gew"ohnliche(s) Differentialgleichung(ssystem) \\
PDE& \enquote{Partial Differential Equation(system)}, Partielle(s) Differentialgleichung(ssystem) \\
TEX& Textsatzsystem zur Erstellung l"angerer Texte und mathematischer Formeln \\
XML& \enquote{Extensible Markup Language}, \enquote{Erweiterbare Auszeichnungssprache} zur Darstellung strukturierter Daten\\
\end{tabular}


%\glsresetall %% all glossary entries should be used in long form (again)

	    %% this is a suggestion: you have to create this file on demand

%% if myaddlistoftodos is set to "true", the current list of open todos is added:
\ifthenelse{\boolean{myaddlistoftodos}}{
  \newpage\listoftodos          %% handy if you are using todonotes with \todo{}
}{}                             %% with todonotes-package option "disable" you can get rid of any todo in the output

\mainmatter                     %% KOMA: marks main part using arabic page numbers and such; only available in scrbook

\cleardoublepage
%% =========================================================================
%%%%%%% Main Part (include your tex file chapters here): %%%%%%%%%%%%
%% =========================================================================
 %% example text content
%% scrartcl and scrreprt starts with section, subsection, subsubsection, ...
%% scrbook starts with part (optional), chapter, section, ...
\chapter{Einleitung/Introduction}

Die in diesem Dokument aufgeführten Richtlinien sollen eine einigermaßen einheitliche Form der am Fachgebiet \enquote{Dynamik und Betrieb technischer Anlagen} (DBTA) angefertigten Abschlussarbeiten gewährleisten.
Sie sind so gewählt, dass die äußere Form den Vorgaben des Prüfungsamts folgt. Zusätzlich wurde ein großer Wert auf eine typographisch ansprechende Formatierung und Seitengestaltung gelegt.

Nur sehr wenige Korrektoren fühlen sich von Rand- und Absatzlosen \enquote{Textwüsten} angezogen. Um einen solchen Eindruck zu vermeiden, werden relativ strenge Vorgaben für Ränder und Zeichen pro Seite gemacht. Auf den ersten Blick mag dies ungewohnt erscheinen, die Werte wurden aber bewusst so gewählt, um eine Ermüdung des Lesers möglichst zu vermeiden. Eine zu große Anzahl an Zeichen pro Zeile bzw. Seite würde auf den Leser \enquote{demotivierend} wirken. Da der Leser von Abschlussarbeiten zunächst oft auch deren Korrektor oder Gutachter ist, wollen wir das zum Besten des Verfassers der Abschlussarbeit ausdrücklich verhindern.

Die Richtlinien sind eindeutig als Hilfestellung zu sehen. Mit diesem Dokument wird außerdem ein Beispiel zur Verfügung gestellt, welches ein Gefühl für den angestrebten Gesamteindruck der anzufertigenden Arbeiten vermittelt.
Bei Nutzung des LaTeX-Codes, der für dieses Dokument verwendet wurde, werden automatisch alle Richtlinien eingehalten und wer auch immer die nächste Abschlussarbeit am Fachgebiet DBTA schreibt, kann sich voll auf den Inhalt konzentrieren, da das typographische \enquote{drumherum} für die gute Präsentation sorgt.\\

Für den Aufbau der Abschlussarbeit sei folgende Struktur gewählt:
\begin{itemize}
	\item Titelseite, die gleichzeitig als \enquote{Schmutztitel} dient
	\item Eidesstattliche Versicherung
	\item (optional) Aufgabenstellung (nur einfügen, falls eine Prüfungsordnung es verlangt)
	\item (optional) Colophon mit Hinweisen zu technischer Herstellung des Dokuments
	\item (optional) Widmung (dedication)
	\item (optional) Danksagung
	\item (optional) Geleitwort (preface) mit Hinweisen zu Vorgeschichte und Entstehungsgeschichte der Arbeit
	\item Inhaltsverzeichnis, welches so früh wie möglich stehen sollte, um dem Leser einen Gesamtüberblick zu vermitteln
	\item Kurzfassung (abstract)
	\item Abbildungsverzeichnis (obligatorisch, sobald es mehrere Abbildungen in der Arbeit gibt)
	\item Tabellenverzeichnis (obligatorisch, sobald es mehrere Tabellen in der Arbeit gibt)
	\item Glossar / Abkürzungsverzeichnis / Liste der Symbole
	\item Hauptteil inklusiver aller Unterkapitel, Beginn der Seitenzahlen mit arabischen Ziffern
	\begin{itemize}
		\item Einleitung, Problemstellung
		\item Theorie, Begriffsklärungen, Stand des Wissens
		\item \dots
		\item Ergebnisse
		\item Diskussion
		\item Schlussfolgerungen
		\item Zusammenfassung
		
	\end{itemize}
	\item (optional) Anhänge inklusive Quellcode, Beweisen, Algorithmen, vertiefenden Beispielen, Datentabellen, die im Hauptteil den Lesefluss unnötig unterbrechen würden
	\item Literaturverzeichnis (bibliography)
\end{itemize}

In den folgenden Kapiteln dieses Dokuments werden die Richtlinien für Abschlussarbeiten aufgelistet (Kapitel 2), für Abschlussarbeiten relevante Normen genannt (Kapitel 3) und eine Anleitung gegeben, wie man eine Abschlussarbeit sinnvollerweise mit LaTeX erstellt (Kapitel 4). Kapitel 5 ist lediglich ein Platzhalter zur Demonstration von Unterkapiteln und der Einbindung von Graphiken. Im Anhang werden hier lediglich im Kapitel \enquote{Quellcode} zwei kleine Java-Dateien gelistet. Zuletzt folgt das Literaturverzeichnis im Name-Jahr-Stil mit Rückreferenzen zu den Vorkommen im Text.\\

Now you are able to write your own document. Always keep in mind: it's
the \emph{content} that matters, not the form. But good typography is
able to deliver the content much better than information set with bad
typography. This template allows you to concentrate on writing good
content while the form is done by the template definitions.


        %% this is a suggestion: you have to create this file on demand
 
\chapter{Richtlinien/Guidelines für Abschlussarbeiten}
\label{cha:guidelines}

\section[Allgemein]{Allgemein}
\begin{enumerate}
	\item Die aktuell geltende Studien- und Prüfungsordnung ist maßgeblich. Folgende Regelungen greifen nur, wenn sie den geltenden Ordnungen nicht widersprechen.
	\item 	Masterarbeiten sind wissenschaftlich-technische Dokumentationen, die allgemeinen Anforderungen bezüglich Struktur und Form entsprechen müssen. Sie sollen sich durch Klarheit im Ausdruck, guten Stil und einwandfreie Orthografie auszeichnen. Formulierungen sind sorgfältig zu wählen.
\item Die Arbeit ist logisch zu gliedern.
\item Die Abschlussarbeit soll einen anspruchsvollen wissenschaftlich-technischen Inhalt verständlich vermitteln. Dazu ist es erforderlich, sich immer wieder in die Position des Lesers zu versetzen und die Darstellung in dieser Hinsicht zu hinterfragen.
\item Das Bild, d.h. Prinzipskizze, Diagramm, Foto, Flussdiagramm, Tabelle etc., ist die „Sprache das Ingenieurs“ und sollte langen Erklärungen vorgezogen werden.
\item Ergebnisse müssen rückverfolgbar sein. Dazu sind die angewendeten Methoden, Annahmen, Randbedingungen, experimentellen Einrichtungen und Programme sowie relevante Zwischenergebnisse zu nennen und in einem angemessenen Umfang zu erläutern.
\item Berechnungen sind so zu dokumentieren, dass der Gutachter ihre Richtigkeit überprüfen kann.
\item Die Darstellung sollte sich auf das Wesentliche konzentrieren und frei von allgemein bekannten Abhandlungen sein, die nur vom Thema ablenken und den „roten Faden“ verlieren lassen.
\end{enumerate}


\section[Form]{Form}
\begin{enumerate}
	\item Die Abschlussarbeit ist in der Regel in deutscher Sprache und dabei nach den Regeln der neuen deutschen Rechtschreibung anzufertigen. 
\item Sätze müssen klar verständlich sein. Schachtelsätze sollten bewusst und nicht zu oft formuliert werden.
\item Der Unterschied zwischen Stichpunkten und Sätzen ist zu beachten.
\item Physikalische Größen sind gemäß DIN 1304-1 und ENISO 80000 in Maßeinheiten des internationalen Einheitensystems (SI) anzugeben. Insbesondere stehen Einheiten niemals in eckigen Klammern.
\item Formelzeichen und Abkürzungen sind in einem gesonderten Verzeichnis in alphabetischer Reihenfolge oder, bei nur einmaliger Benutzung, im Text zu erläutern. Die physikalischen Einheiten sind neben Symbol und Bedeutung in einer separaten Spalte des Symbolverzeichnisses aufzuführen.
\item Bilder, Tabellen und Formeln sind zu nummerieren und müssen einen Textbezug haben. Bilder werden mit einer Bildunterschrift versehen.
\item Bilder, die in den Text eingefügt werden, sollten sorgfältig ausgewählt werden, um Textfluss und Lesbarkeit zu unterstützen und nicht zu beeinträchtigen. Die wesentlichen Details des Bildes und relevante Beschriftungen müssen in allen abgegebenen Exemplaren gut erkennbar sein. 
\item Gleichungen sind fortlaufend je Kapitel zu nummerieren, die Nummer des Kapitels wird, durch einen Punkt abgetrennt, vorangestellt. Die Gleichungsnummer ist in runden Klammern anzugeben und rechtsbündig zu platzieren. Beispielsweise ist (3.1) die erste Gleichung in Kapitel 3, wohingegen (B.13) die dreizehnte Gleichung in Anhang B ist.
\item Erläuternde Beispiele, große Bilder und ergänzende Daten sowie Beilagen, auf die im Text immer wieder hingewiesen wird, gehören in die Anlage. Längere Dokumente, die der Verständlichkeit dienen, Programmquelltexte, längere Beispiele o.ä. sind als Anhang in gedruckter oder elektronischer Form beizufügen.
\item Enthält die Arbeit Zitate, so müssen diese entsprechend ausgewiesen und die Quellen nachgewiesen werden. Es muss klar ersichtlich sein, welche Erkenntnisse vom Autor stammen und welche der Literatur entnommen wurden. Alle benutzten Quellen, auch für sinngemäß wiedergegebene Gedanken, Erfahrungswerte usw. sind im Text zu referenzieren und im Literaturverzeichnis aufzuführen. Wörtliche Zitate sind durch Anführungszeichen zu kennzeichnen.
\end{enumerate}


\section[Formatierung]{Formatierung}
\begin{enumerate}
	\item Die Arbeit ist in einer proportionalen Serifenschrift zu verfassen, \zB Palatino in LaTeX (mathpazo). Überschriften dürfen, wenn die Arbeit mit LaTeX erstellt wird, serifenlos sein.
\item Der Text ist einseitig im Format A4 in Blocksatz und mit einer Schriftgröße von mindestens 11 pt zu setzen. 
\item Das Textfeld (\enquote{Satzspiegel}) muss eine Breite zwischen 130 mm und 155 mm haben, wodurch sich Ränder von 30-40 mm je Seite ergeben. In der ausgedruckten Version ist eine angemessene Bindekorrektur einzuplanen, die den Satzspiegel um rund 5-10 mm nach rechts verschiebt (und somit den linken Rand gegenüber dem rechten Rand vergrößert).
\item Der Satzspiegel nimmt in der Höhe zwischen 6/9 und 8/11 der Seite ein, wobei der Abstand zum unteren Rand der Seite doppelt so groß ist wie der Abstand zum oberen Rand ist. Das entspricht Rändern von 27 - 33 mm (oben) und 54 - 66 mm (unten).
\item Die Seitenzahl befindet sich zentriert unter dem Satzspiegel mit mindestens zwei Zeilen Abstand von diesem.
\item Über dem Satzspiegel werden, mit mindestens einer Zeile Abstand, in serifenloser Schrift die Kapitelnummer und die Kapitelüberschrift in gleicher Schriftgröße wie der Haupttext platziert, z.B. \enquote{4. Ergebnisse}. Diese \enquote{Erinnerungsüberschriften} dürfen durch eine durchgezogene Linie vom Satzspiegel optisch abgehoben werden.
\item Überschriften sind genauso wie Legenden zu Abbildungen sowie Tabellen linksbündig zu setzen. Der Abstand vor/über einer Überschrift ist 1,5- bis 2-mal so groß wie danach/darunter (zu Text oder Unterüberschrift).
\item In Bildunterschriften etc. ist eine kleinere Schriftgröße zu wählen (z.B. 9 pt statt 11 pt).
\item Jede Seite im Hauptteil der Arbeit hat als Daumenregel maximal 74 Zeichen pro Zeile und 34 Zeilen pro Seite. Das entspricht maximal rund 2500 Zeichen pro Seite.
[In LaTeX (KOMA-Script) ist für die Schrift Palatino ein linespread von 1.2 vorzusehen (Spielraum: 1.0 bis 1.5). Das kann in anderen Sprachen einem Zeilenabstand von 1,2 bis 1,5 entsprechen. Die maximale Anzahl an Zeilen gibt hier eine Orientierung.]
\item Der gesamte Hauptteil der Arbeit soll nicht mehr als 220.000 Zeichen enthalten (das entspricht knapp 90 Seiten). Wird diese Grenze überschritten, ist gemeinsam mit dem Betreuer zu prüfen, ob eine Kürzung angebracht/erforderlich ist. 
\end{enumerate}


\section[Erscheinungsbild]{Äußeres Erscheinungsbild}
\begin{enumerate}
	\item Im Hauptteil (\enquote{mainmatter}) beginnt die Zählung der Seiten mit arabischen Ziffern bei 1, vorher (\enquote{frontmatter}) sind römische Ziffern zu verwenden.
\item Es sind maximal vier Gliederungsebenen zu verwenden und maximal drei Ebenen im Inhaltsverzeichnis zu notieren.
\item 	Die einzelnen Abschnitte des Anhangs sind nicht numerisch, sondern durch Großbuchstaben zu kennzeichnen.
\item Zur Hervorhebung im Text sind bevorzugt Fett- oder Kursivschreibung zu verwenden, Unterstreichungen sind nicht erwünscht.
\item Absätze, die auf zwei Seiten verteilt sind, dürfen nicht in der letzten oder vorletzten Zeile einer Seite beginnen (\enquote{Schusterjunge}) oder in der ersten oder zweiten Zeile einer Seite enden (\enquote{Hurenkinder}).
\end{enumerate}


\section[Quellenangaben und Literaturverzeichnis]{Quellenangaben und Literaturverzeichnis}
Zitate/Quellenangaben dienen dazu,
\begin{itemize}
	\item eigene Aussagen zu belegen und zu begründen,
	\item eigene Aussagen von fremden zu unterscheiden,
	\item den Lesern die Quellen der Aussagen rasch nachvollziehbar zu machen.
\end{itemize}
Alle fremden Inhalte werden im Text mit einem kurzen Quellenhinweis gekennzeichnet, der durch eine ausführlichere Quellenangabe im Literaturverzeichnis ergänzt wird. Dabei spielt es keine Rolle, ob die fremden Inhalte wörtlich/direkt oder sinngemäß/indirekt übernommen werden.\\
Jede Quellenangabe im Text weist auf eine Angabe im Literaturverzeichnis hin und jede Quellenangabe im Literaturverzeichnis ist die Erläuterung (mindestens) eines Kurzhinweises im Text. Wenn die Bibliographie mit LaTeX erstellt wird, ergibt sich das quasi von selbst. Es ist erlaubt, im Quellenverzeichnis die Seiten der Vorkommen im Text zu nennen und Referenzen (Links) zu setzen.\\
Für Abschlussarbeiten verwenden wir bevorzugt das \enquote{Namen-Datum-/Autor-Jahr-System}, auf das sich die folgenden Erläuterungen und Beispiele beziehen; das \enquote{Nummernsystem} ist aber auch möglich. Bedeutendste Grundregel von allen: Vollständige und einheitliche Angaben sind wichtiger als das Einhalten eines speziellen Standards. \\
Die Mindestanforderungen sind: Autor, Titel, Datum/Jahr.

\subsection{Zitieren im Literaturverzeichnis}

Wenn LaTeX zur Erstellung der Abschlussarbeit verwendet wird, ist bevorzugt der Biblatex-Stil \enquote{authoryear-comp} (oder \enquote{authoryear}) zu wählen. \textit{Alle folgenden Angaben sind also nur dann relevant, wenn das Literaturverzeichnis von Hand erstellt wird.}\\

\subsubsection{Literaturverzeichnis von Hand anlegen}
Die Quellen werden alphabetisch nach dem Nachnamen ihrer Verfasser sortiert. Sollten mehrere Werke von denselben Autoren stammen, so werden diese Quellenangaben chronologisch gereiht (von alten nach jungen Veröffentlichungen). Wenn dies noch nicht eindeutig ist, wird die Jahreszahl zusätzlich mit einem Kleinbuchstaben versehen.
Innerhalb der einzelnen Literaturangaben wird die Reihenfolge der Autoren unverändert aus der Originalquelle entnommen. Die ersten sechs Autoren werden immer angegeben; alle weiteren Autorennamen werden durch \enquote{et al.} ersetzt. Bei zwei bis sechs Autoren darf vor dem letzten Autor ein \enquote{und}, \enquote{and} oder \enquote{\&} stehen.
Die Auflistung der Autoren wird durch einen Punkt abgeschlossen; danach folgt, durch ein Leerzeichen getrennt, das Jahr der Veröffentlichung (und ggflls. ein Kleinbuchstabe, mit oder ohne Leerzeichen von der Jahreszahl getrennt).
Nach dem Erscheinungsjahr folgen ein Doppelpunkt, ein Leerzeichen und der vollständige Haupttitel der Quelle (kursiv). Der Titel wird wiederum mit einem Punkt abgeschlossen.
Es folgen, je nach Art der Quelle, Angaben zu 
\begin{itemize}
	\item Zeitschrift/Sammelband (\enquote{In:}), 
	\item Verlag (\enquote{Burg-Verlag}), 
	\item Ort der Veröffentlichung (\enquote{Berlin}), 
	\item Jahrgang des Bandes, Heftnummer
	\item Seitenangaben (\enquote{182-188}),
	\item ISSN/ISBN/doi-Nummern (\enquote{ISSN 0815-0008})
\end{itemize}
Bei Artikeln in Zeitschriften oder in Sammelbänden müssen die Seitenzahlen angegeben werden.
Bei Büchern sollte neben dem Ort auch der Verlag angegeben werden. Bei Zeitschriften werden Verlag und Ort in der Regel nicht angegeben.
Wichtig ist die Einheitlichkeit der Form bei allen Angaben und Satzzeichen.
Die Nachnamen werden in Großbuchstaben geschrieben, von den Vornamen wird jeweils nur der erste Buchstabe genannt (ohne Abkürzungspunkt). 
Zwischen Nachname und Vorname steht ein einzelnes Leerzeichen, kein Komma.
Mehrere Verfasser werden mit Komma getrennt.
             %% this is a suggestion: you have to create this file on demand
 %% example text content
%% scrartcl and scrreprt starts with section, subsection, subsubsection, ...
%% scrbook starts with part (optional), chapter, section, ...
\chapter{Relevante Normen}
\begin{itemize}
	\item DIN 1421 (ISO 2145)
	\begin{itemize}
\item Titel: Gliederung und Benummerung in Texten; Abschnitte, Absätze, Aufzählungen
\item Beschreibung: Gestaltung von Texten, um diese in Abschnitte, Absätze oder Aufzählungen zu gliedern 
\end{itemize}

\item DIN1422-1 und DIN 1422-4
\begin{itemize}
\item (Veröffentlichungen aus Wissenschaft, Technik, Wirtschaft und Verwaltung)
\item Teil 1, Gestaltung von Manuskripten und Typoskripten
\item Teil 4, Gestaltung von Forschungsberichten
\end{itemize}

\item DIN 1426 
\begin{itemize}
\item Inhaltsangaben von Dokumenten; Kurzreferate, Literaturberichte
\end{itemize}

\item DIN ISO 690:2013-10 (ersetzt DIN 1505)
\begin{itemize}
\item Richtlinien für Titelangaben und Zitierung von Informationsressourcen
\end{itemize}

\item DIN 5008
\begin{itemize}
\item Schreib- und Gestaltungsregeln für die Textverarbeitung
\end{itemize}

\item DIN 1304-1 
\begin{itemize}
\item Formelzeichen; Allgemeine Formelzeichen
\end{itemize}

\item EN ISO 80000
\begin{itemize}
\item Größen und Einheiten
\end{itemize}
\end{itemize}



             %% this is a suggestion: you have to create this file on demand
 %% example text content
%% scrartcl and scrreprt starts with section, subsection, subsubsection, ...
%% scrbook starts with part (optional), chapter, section, ...
\chapter{Abschlussarbeit mit LaTeX erstellen}
\section{TeX-Distributionen}

Empfohlene Tex-Distributionen:
\begin{itemize}
\item TeX Live, 
\item MiKTeX (Windows) 
\end{itemize}


\subsection{TeX Live}
TeX Live ist die wohl vollständigste Tex-Distribution und ist auf allen Plattformen verfügbar.
\subsection{MiKTeX}
MiKTeX ist eine TeX-Distribution, die auf Windows-Betriebssystem verfügbar ist. Der Download kann direkt über die Homepage erfolgen:

https://miktex.org/download\\

Ende Juni 2017 ist Version 2.9.6 die aktuelle Version von MiKTeX.
Basic MiKTeX kann so eingestellt werden, dass benötigte Pakete später nachgeladen werden. 
Bei der Auswahl \enquote{Complete MiKTeX} werden im Sinne einer Komplettinstallation alle Pakete auf einmal geladen und installiert.
Die Gesamtgröße des Pakets beträgt rund 2,4 GB und das Herunterladen kann dementsprechend 30-40 Minuten dauern.
Die Initialisierung der Installation hat bereits 15 bis 20 Minuten in Anspruch genommen. Die tatsächliche Installation braucht auch einige Minuten, sodass mit insgesamt mehr als 30 Minuten zu rechnen ist; also Geduld haben und nicht gleich abbrechen, auch wenn sich der Fortschrittsbalken mal nicht rührt.\\

Zusätzliche Herausforderung: Das Paket Asymptote wird \textit{nicht} mit MiKTeX mitgeliefert und muss bei Bedarf manuell installiert werden, zum Beispiel, indem es mit Hilfe eines lokalen texmf-Baums aus einem Verzeichnis gemäß  der TeX Directory Structure (TDS) eingebunden wird. Beschreibung und Anleitungen gibt es natürlich einige im Internet, hier zwei Links zum Einstieg in die Thematik:
\begin{itemize}
	\item https://tex.stackexchange.com/questions/83035/using-asymptote-with-miktex
	\item https://tex.stackexchange.com/questions/69483/create-a-local-texmf-tree-in-miktex
\end{itemize}


Bei den MiKTeX Settings (aufzurufen z.B. über das Windows-Startmenü) kann es ausreichen, die Nicht-Admin-Version zu verwenden und dort einen root-path einzugeben (+ Refresh FNDB)

\section{TeX-Entwicklungsumgebungen}

TeXstudio, Texmaker, TeXnicCenter (Windows), TeXShop (macOS)
Es ist Geschmacks- und Gewöhnungssache, mit welcher Entwicklungsumgebung man am produktivsten arbeitet, deshalb wird hier keine Empfehlung für einen bestimmten Editor ausgesprochen.

\section{KOMA-Script}
Das in der Latex-Vorlage verwendete KOMA-Script ist ausführlich in der Datei scrguide dokumentiert.

Für Abschlussarbeiten, die nur am Fachgebiet DBTA angefertigt werden, ist die in der Latex-Vorlage vorbereitete Titelseite zu verwenden. Sollten mehrere Lehrstühle oder Kooperationspartner aus der Industrie beteiligt sein, kann diese Titelseite von der Auswahl der Logos abgewandelt werden. Es kann auch komplett auf Logos verzichtet werden.


\section{Literaturverwaltung, Bibliographie, Referenzen}

Wer schon gute Erfahrung mit bibtex und natbib hat, kann dabei bleiben, in allen anderen Fällen wird empfohlen, die aktuelleren/moderneren Pakete/Programme biber und biblatex zu verwenden. Mit den Literaturverwaltungsprogrammen Citavi oder JabRef oder Mendeley lassen sich Einträge für die \enquote{.bib}-Datei erzeugen, die letztendlich von biber verarbeitet und mit biblatex im Quellcode eingebunden wird. Da biber und biblatex für eine Zusammenarbeit konzipiert sind, ist bitte darauf zu achten, dass die Versionen von biber und biblatex zueinander passen / aktuell sind.\\

Hier passende Hinweise aus der LaTeX-Community:

\begin{itemize}
	\item http://golatex.de/wichtige-hinweise-erstellung-von-literaturverzeichnissen-t11964.html
\end{itemize}

Hinweise zur Einbindung von biber in den jeweiligen Editor:

\begin{itemize}
	\item http://www.texwelt.de/wissen/fragen/1909/wie-verwende-ich-biber-in-meinem-editor
\end{itemize}

Zum Einlesen in die Unterschiede hier ein Link von einer Diskussion, die im Jahr 2011 begann:

\begin{itemize}
	\item https://tex.stackexchange.com/questions/25701/bibtex-vs-biber-and-biblatex-vs-natbib
\end{itemize}

Bei allen bib- und tex-Dateien ist darauf zu achten, dass durchgehend eine UTF-8 Kodierung verwendet wird. Bei TeXstudio und Notepad++ ist bestätigt, dass dies direkt einstellbar ist. Bei Verwendung anderer Editor ist dies entsprechend zu überprüfen.



             %% this is a suggestion: you have to create this file on demand
 \chapter{Next Example Chapter}

\section{first section of example chapter}
Bla blabla bla.
This is my own text with an example Figure~\ref{fig:example} and example
citation~\cite{StrunkWhite} or \textcite{Bringhurst1993}. And there is another
\enquote{citation} which is located at the bottom\footcite{tagstore}.

\myfig{TUBerlin_Logo_rot}%% filename in figures folder
  {width=0.1\textwidth,height=0.1\textheight}%% maximum width/height, aspect ratio will be kept
  {Example figure 2.}%% caption
  {Another example figure}%% optional (short) caption for table of figures
  {fig:example2}%% label

Now you are able to write your own document. Always keep in mind: it's
the \emph{content} that matters, not the form. But good typography is
able to deliver the content much better than information set with bad
typography. This template allows you to concentrate on writing good
content while the form is done by the template definitions.

Now you are able to write your own document. Always keep in mind: it's
the \emph{content} that matters, not the form. But good typography is
able to deliver the content much better than information set with bad
typography. But good typography is able to deliver the content much better than information set with bad
typography. But good typography is able to deliver the content much better than information set with bad
typography. This template allows you to concentrate on writing good
content while the form is done by the template definitions.

Now you are able to write your own document. Always keep in mind: it's
the \emph{content} that matters, not the form. But good typography is
able to deliver the content much better than information set with bad
typography. This template allows you to concentrate on writing good
content while the form is done by the template definitions.

Now you are able to write your own document. Always keep in mind: it's
the \emph{content} that matters, not the form. But good typography is
able to deliver the content much better than information set with bad
typography. This template allows you to concentrate on writing good
content while the form is done by the template definitions. But good typography is able to deliver the content much better than information set with bad
typography. This template allows you to concentrate on writing good
content while the form is done by the template definitions.

\section{second section of example chapter}

\subsection{This is the subsection}
Ein subsection ist das niedrigste Gliederungselement, welches noch im Inhaltsverzeichnis aufgeführt wird (sozusagen \enquote{Level 3}).
A subsection is the lowest level of section that is mentioned in the table of contents.\\

Now you are able to write your own document. Always keep in mind: it's
the \emph{content} that matters, not the form. But good typography is
able to deliver the content much better than information set with bad
typography. This template allows you to concentrate on writing good
content while the form is done by the template definitions. But good typography is able to deliver the content much better than information set with bad
typography. This template allows you to concentrate on writing good
content while the form is done by the template definitions.

Now you are able to write your own document. Always keep in mind: it's
the \emph{content} that matters, not the form. But good typography is
able to deliver the content much better than information set with bad
typography. This template allows you to concentrate on writing good
content while the form is done by the template definitions. But good typography is able to deliver the content much better than information set with bad
typography. This template allows you to concentrate on writing good
content while the form is done by the template definitions.

Now you are able to write your own document. Always keep in mind: it's
the \emph{content} that matters, not the form. But good typography is
able to deliver the content much better than information set with bad
typography. This template allows you to concentrate on writing good
content while the form is done by the template definitions. But good typography is able to deliver the content much better than information set with bad
typography. This template allows you to concentrate on writing good
content while the form is done by the template definitions.

Now you are able to write your own document. Always keep in mind: it's
the \emph{content} that matters, not the form. But good typography is
able to deliver the content much better than information set with bad
typography. This template allows you to concentrate on writing good
content while the form is done by the template definitions. But good typography is able to deliver the content much better than information set with bad
typography. This template allows you to concentrate on writing good
content while the form is done by the template definitions.             %% this is a suggestion: you have to create this file on demand
% \include{chapters/solution}            %% this is a suggestion: you have to create this file on demand
% %%%% Time-stamp: <2013-02-25 10:31:01 vk>


\chapter*{Evaluation}
\label{cha:evaluation}


This is a placeholder for the evaluation.



%\glsresetall %% all glossary entries should be used in long form (again)
%% vim:foldmethod=expr
%% vim:fde=getline(v\:lnum)=~'^%%%%\ .\\+'?'>1'\:'='
%%% Local Variables:
%%% mode: latex
%%% mode: auto-fill
%%% mode: flyspell
%%% eval: (ispell-change-dictionary "en_US")
%%% TeX-master: "main"
%%% End:
          %% this is a suggestion: you have to create this file on demand
% \include{chapters/outlook}             %% this is a suggestion: you have to create this file on demand
% \include{chapters/summary}             %% this is a suggestion: you have to create this file on demand

%% =========================================================================
%%%%%%%% Appendix (appendix with prefix e.g. 'A Quellcode') %%%%%%%%%%%%%%%%%%%%%
%%%%%%%% include your source code etc. here:                %%%%%%%%%%%%%%%%%%%%%
%% =========================================================================

\appendix                       %% closes main document, appendix follows until end; only available in book-classes
%\addpart{Appendix}             %% adding Appendix to tableofcontents
\chapter{Quellcode}
\renewcommand{\lstlistlistingname}{\mylistingstitle}	%% set title of the list of listings
\lstlistoflistings							%% create list of listings

\newpage
\lstset{basicstyle=\tiny,language=Java,breaklines=true,keywordstyle=\color{blue!80!black!100},commentstyle=\color{green!50!black!100}, numbers=left}
\lstinputlisting[caption=Example1.java]{code/Example1.java}
\newpage
\lstinputlisting[caption=Example2.java]{code/Example2.java}

  %% <---- this is a suggestion: you have to create this file on demand


\backmatter											%% appendix without prefix, e.g. 'Bibliography'

\renewcommand{\bibname}{\mybibliographytitle} %% change title of the bibliography
\printbibliography              %% print bibliography with biblatex


%% =========================================================================
%%%%%%%% Further Ressources (include your further ressources / glossary here): %%%%%%%%%%%%%
%% =========================================================================
% \include{chapters/further_ressources}  %% this is a suggestion: you have to create this file on demand



%%%% end{document}
\end{document}
